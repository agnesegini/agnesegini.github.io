\documentclass[a4paper,11pt,oneside]{book}

\usepackage[italian]{babel}
\usepackage[T1]{fontenc}
\usepackage[utf8]{inputenc}
\usepackage{amsmath}
\usepackage{amsfonts}
\usepackage{amssymb}
\usepackage{amstext}
\usepackage{amscd}
\usepackage{amsthm}
\usepackage{faktor}
\usepackage{mathtools}
\usepackage{hyperref}
\hypersetup{colorlinks=blue, pdfborder={0 0 0}}
\usepackage{tikz}
\usepackage[all]{xy}
\usetikzlibrary{arrows,chains,matrix,positioning,scopes}
\usepackage{enumerate}
  
%abbreviazioni che possono servire
\newcommand{\Spec}{\mathrm{Spec}}
\newcommand{\ideals}{\mathcal{I}} 
\newcommand{\N}{\mathbb{N}}
\newcommand{\Z}{\mathbb{Z}}
\newcommand{\C}{\mathcal{C}}
\newcommand{\inverti}[1]{{#1}^{-1}}
\newcommand{\A}{\mathcal{A}}
\newcommand{\B}{\mathcal{B}}
\newcommand{\m}{\mathfrak{m}}
\newcommand{\set}[1]{\left\{ #1 \right\}}
\newcommand{\der}[1]{\mathcal{D}(#1)}
\newcommand{\complx}[1]{#1^{^\bullet}}
\renewcommand{\dh}{\mathrm{dh}}

% TRIANGOLO
% \begin{center}
% 	       \begin{tikzpicture} [scale=1.8]
% 		 \node (a) at (-1,0) {$\complx{#1}$};
% 		 \node (b) at (0,0) {$\complx{#2}$};
% 		 \node (c) at (1,0) {$\complx{#3}$};
% 		 \node (d) at (2,0) {$\complx{#1}[1]$};
% 		 \draw[->] (a) to node[above]{$#4$} (b);
% 		 \draw[->] (b) to node[above]{$#5$} (c);
% 		 \draw[->] (c) to node[above]{$#6$} (d);
% 		\end{tikzpicture}
% \end{center}


\DeclareMathOperator{\Ob}{Ob}
\DeclareMathOperator{\Mor}{Mor}
\DeclareMathOperator{\Hom}{Hom}
\DeclareMathOperator{\Kom}{Kom}
\DeclareMathOperator{\Com}{Com}
\DeclareMathOperator{\Cil}{Cil}
\DeclareMathOperator{\Cono}{Cono}
\DeclareMathOperator{\Ext}{Ext}
\DeclareMathOperator{\Tor}{Tor}
\DeclareMathOperator{\Imm}{Im}
\DeclareMathOperator{\rank}{rk}
\DeclareMathOperator{\dhp}{dhp}
\DeclareMathOperator{\dhi}{dhi}
\DeclareMathOperator{\tors}{Tors}
\DeclareMathOperator{\Ann}{Ann}


\theoremstyle{definition}
\newtheorem{de}{Definizione}
\newtheorem{teo}[section]{Teorema}
\newtheorem*{passo}{Passo}
\newtheorem{prop}{Proposizione}
\newtheorem{lemma}{Lemma}
\newtheorem{cor}{Corollario}
\newtheorem{notz}{Notazione}


\newtheorem{esempio}{Esempio}
\newtheorem{esem}[section]{Esempio}
\newtheorem{oss}[section]{Osservazione}


\title{\textbf{Appunti del corso di Istituzioni di Algebra 2015/2016}\\ \textsc{Funtori Derivati e Applicazioni dell'Algebra Omologica}}
\author{Agnese Gini}


\begin{document}
\maketitle
\tableofcontents
\chapter{Moduli localmente liberi}
 \begin{teo}\label{teo:loclibero}
  Sia $A$ un anello noetheriano e $M$ un $A$ modulo finitamente generato. Allora sono equivalenti i seguenti fatti:
   \begin{enumerate}[a.]
    \item $M$ proiettivo;
    \item $M_p$ libero per ogni $p$ ideale primo di $A$;
    \item $M_{\m}$ libero per ogni $\m$ ideale massimale di $A$;
    \item esistono $a_1,\dots,a_n\in A$ tali che l'ideale $(a_1,\dots,a_n)=A$ e 
	  $M_{a_i}$ è un $A_{a_i}$ modulo libero per $i=1,\cdots,n$.
   \end{enumerate}
 \end{teo}

 \begin{de}
  Se valgono a, b e c $M$, allora si dice \emph{localmente libero}.
 \end{de}

 \begin{lemma}\label{loc:proiettivoimplicalibero}
  Sia $A$ un anello locale noetheriano e $M$ un $A$ modulo proiettivo finitamente generato. Allora $M$ è libero.
 \end{lemma}

 \begin{proof}
  Indichiamo con $\m$ l'ideale massimale di $A$. Dato che il quoziente $\faktor{M}{\m M}$ è uno spazio vettoriale su $k=\faktor{A}{\m}$,
  possiamo prendere $\bar x_1,\dots,\bar x_n$ una sua base con $ x_i\in M$. Vogliamo mostrare che sono una base per $M$ come $A$ modulo.
  Per il lemma di Nakayama $ x_1,\dots, x_n$ generano $M$; prendiamo adesso il morfismo 
  \[\begin{matrix}
   f\colon& A^n & \twoheadrightarrow & M\\
   & e_i & \mapsto & x_i
  \end{matrix}.
  \]
  se proviamo che è un isomorfismo, in particolare che $N\coloneqq\ker f=0$ abbiamo finito.
  $M$ è proiettivo e dunque la successione
    \begin{center}
	\begin{tikzpicture}[scale=1]
	 \node (a) at (-1,0) {$0$};
	 \node (b) at (0,0) {$N$};
	 \node (c) at (1,0) {$A^n$};
	 \node (d) at (2,0) {$M$};
	 \node (a3) at (3,0) {$0$};
	 \draw[->] (a) -- (b);
	 \draw[->] (b) to node[above]{$ $} (c);
	 \draw[->] (c) to node[above]{$f$} (d);
	 \draw[->] (d) -- (a3);
	\end{tikzpicture}
    \end{center} 
    spezza, ossia $A^n\simeq N\oplus M$. Quozientando per $\m$ otteniamo che
    \[
     \faktor{A^n}{\m A^n}\simeq \faktor{N}{\m N}\oplus \faktor{M}{\m M}
    \]
    Osserviamo che ${A^n}/{\m A^n}=k^n$ e dato che passando $f$ a quoziente otteniamo un omomorfismo suriettivo, dovremmo
    necessariamente avere che ${N}/{\m N}=0$ che implica $N=0$.
 \end{proof}

 \begin{lemma}\label{lemma:3implica4}
  Sia $A$ un anello noetheriano e $p$ un suo ideale primo, $M$ e $N$ due $A$ moduli finitamente generati. Se
  $M_p\simeq N_p$ allora esiste $a\notin p$ tale che $M_a\simeq N_a$.
 \end{lemma}
 \begin{proof}
  Consideriamo le seguenti successioni 
  \[\begin{matrix}
   A^h&\xrightarrow{\alpha}& A^n & \twoheadrightarrow& M\\
      & 		    & e_i & \mapsto & m_i
  \end{matrix}.
  \]
  \[\begin{matrix}
   A^k&\xrightarrow{\beta}& A^n & \twoheadrightarrow & N\\
      & 		    & e_i & \mapsto & n_i
  \end{matrix}.
  \]
  Localizzando con $p$ si ha
  \begin{center}
      \begin{tikzpicture}[scale=1.8]
		 \node (a) at (-1,0) {$A_p^h$};
		 \node (b) at (0,0) {$A_p^m$};
		 \node (c) at (1,0) {$M_p$};
		 \node (d) at (2,0) {$0$};
		 \node (2a) at (-1,-1) {$A_p^k$};
		 \node (2b) at (0,-1) {$A_p^n$};
		 \node (2c) at (1,-1) {$N_p$};
		 \node (2d) at (2,-1) {$0$};
		 \draw[->] (a) to node[above]{$ $} (b);
		 \draw[->] (b) to node[above]{$ $}  (c);
		 \draw[->] (c) -- (d);
		 \draw[->] (2a) -- (2b);
		 \draw[->] (2b) -- (2c);
		 \draw[->] (2c) -- (2d);
		 \draw[->] (a) to node[left]{$\varphi_2$} (2a);
		 \draw[->] (b) to node[left]{$\varphi_1$} (2b);
		 \draw[->] (c) to node[left]{$\varphi_0$} (2c);
		 \draw[->] (d) to node[left]{$0$} (2d);
		 \end{tikzpicture}
    \end{center}
   dove $\varphi_0$ è un isomorfismo con in inverso $\psi_0$. Esiste $s\notin p$ tale che la matrice $n\times m$ che rappresenta $\varphi_1$ è
    \[
     \left( \frac{a_{ij}}{s} \right)_{ i=1\dots n \, j=1\dots m}
    \]
    con $a_{ij}\in A$. 
    In particolare $\varphi_1$ è una mappa tra $A_s^m$ e $A_s^n$; vogliamo mostrare ora che, a meno di sostituire opportunamente $s$,
    esistono anche $\tilde \varphi_2$ e $\tilde \varphi_0$ e che quest'ultima ha un'inversa $\tilde\psi_0$.
    \begin{center}
      \begin{tikzpicture}[scale=1.8]
		 \node (a) at (-1,0) {$A_s^h$};
		 \node (b) at (0,0) {$A_s^m$};
		 \node (c) at (1,0) {$M_s$};
		 \node (d) at (2,0) {$0$};
		 \node (2a) at (-1,-1) {$A_s^k$};
		 \node (2b) at (0,-1) {$A_s^n$};
		 \node (2c) at (1,-1) {$N_s$};
		 \node (2d) at (2,-1) {$0$};
		 \draw[->] (a) to node[above]{$\alpha_s $} (b);
		 \draw[->] (b) to node[above]{$f_s$}  (c);
		 \draw[->] (c) -- (d);
		 \draw[->] (2a) to node[above]{$\beta_s $} (2b);
		 \draw[->] (2b) to node[above]{$g_s$} (2c);
		 \draw[->] (2c) -- (2d);
		 \draw[->,dashed] (a) to node[left]{$\tilde \varphi_2$} (2a);
		 \draw[->] (b) to node[left]{$\varphi_1$} (2b);
		 \draw[->,dashed] (c) to node[left]{$\tilde\varphi_0$} (2c);
		 \end{tikzpicture}
    \end{center}
    Per provare l'esistenza di $\tilde \varphi_2$ mi basta fare vedere che $\varphi_1\circ \alpha (e_i)\in\Imm \beta_s$;
    ma esiste $t\notin p$ tale che $\varphi_1\circ \alpha (e_i)\in \Imm \beta / t$. Sostituendo $s$ con $st$ abbiamo l'appartenenza voluta.\\
    Definiamo adesso $\tilde\varphi_0(m_i)=g_s\varphi_1(e_i)$; noi vorremmo che $(\tilde\varphi_0)_p=\varphi_0$, ma questo discende 
    dalla seguente catena di uguaglianze:
    \[
     \tilde\varphi_0(m_i)=\tilde\varphi_0(f_s(e_i))=g_s\varphi_1(e_i)=g\varphi_1(e_i)=\varphi_0f(e_i)=\varphi_0(m_i)
    \]
    Analogamente possiamo costruire $\tilde\psi_0$.\\
    Dobbiamo mostrare che è l'inverso. Abbiamo che $(\tilde\psi_0\circ\tilde\varphi_0)_p=id$ e 
    $(\tilde\varphi_0\circ\tilde\psi_0)_p=id$, allora $\tilde\psi_0\circ\tilde\varphi_0 (m_i)-m_i=0$ in $M_p$ e
    $\tilde\varphi_0\circ\tilde\psi_0 (m_i)-m_i=0$ in $N_p$, cioè esiste $u\notin p$ tale che $u(\tilde\psi_0\circ\tilde
    \varphi_0 (m_i)-m_i)=0$ e
    $u(\tilde\varphi_0\circ\tilde\psi_0 (m_i)-m_i)=0$. Sostituendo $s$ con $su$ abbiamo che
    $\tilde\psi_0\circ\tilde\varphi_0=id$ e $\tilde\varphi_0\circ\tilde\psi_0=id$ e dunque $M_s=N_s$.
 \end{proof}
 
  Dimostriamo, usando questi due lemmi, il Teorema \ref{teo:loclibero}:
 \begin{proof}\
 \begin{itemize}
  \item [a. $\Rightarrow$ b.] 
	Consideriamo $p\in\Spec A$, se $M$ è proiettivo allora lo è anche $M_p$, allora per il Lemma \ref{loc:proiettivoimplicalibero}
	dato che $A_p$ è locale si ha la tesi.
  \item [b. $\Rightarrow$ c.] ovvio.
  \item [c. $\Rightarrow$ b.] Sia $p\in\Spec A$ e $\m$ un massimale che lo contiene; indichiamo con $S=A\setminus p$ e $R=A\setminus \m$.
	Per ipotesi $M_{\m}$ è libero, allora $M_p=\inverti{S}M=\inverti{S}M_{\m}$ e dunque $M_p$ è libero.
  \item [d. $\Rightarrow$ c.] Sia $\m$ un massimale e $a_i\notin\m $ e $R=A\setminus \m$. Allora $\inverti{R}M_{a_i}$ e quindi $M_{\m}$
	  è libero.
  \item [c. $\Rightarrow$ d.]  Per ogni $\m$ si ha $M_{\m}\simeq(A^n)_{\m}$, allora per il Lemma \ref{lemma:3implica4}
	  esiste $s_{\m}\notin\m$ tale che $M_{s_\m}\simeq(A^n)_{s_\m}$. Consideriamo l'ideale
	  $I\coloneqq(s_{\m}|\m\in\Spec A \text{ massimale })$, per costruzione $I$ non è contenuto in nessun massimale e quindi $I=A$.
	  Per noetherianità esiste un insieme finito di generatori $a_1,\dots, a_k$, che sono gli elementi che cercavamo.	  
  \item [c. $\Rightarrow$ a.] Facciamo prima una piccola osservazione

 
 \begin{oss}\
 \begin{itemize}
  \item Se $M$ e $N$ sono due moduli su $A$, anello noetheriano, ed esiste $f\colon A\rightarrow B$ piatta su $A$, allora
	  \[
	  B\otimes_A\Hom_A(M,N)=_{\Phi}\Hom_B(B\otimes_A M,B\otimes_A N)
	  \]
	  dove $b\otimes \Phi(\beta \otimes m)=b\beta \Phi (m)$.\\
	  Se $M=A^n$ allora
	  \[
	  B\otimes_A\Hom_A(A^n,N)\simeq B\otimes N^n=_{\Phi}(N\otimes_A B)^n
	  \]
  \item In generale 
	\begin{center}
      \begin{tikzpicture}[scale=1.8]
		 \node (a) at (-1,0) {$A^h$};
		 \node (b) at (0,0) {$A^n$};
		 \node (c) at (1,0) {$M$};
		 \node (d) at (2,0) {$0$};
		 \node (2a) at (-1,-1) {$B^h$};
		 \node (2b) at (0,-1) {$B^n$};
		 \node (2c) at (1,-1) {$M\otimes B$};
		 \node (2d) at (2,-1) {$0$};
		 \draw[->] (a) to node[above]{$ $} (b);
		 \draw[->] (b) to node[above]{$ $}  (c);
		 \draw[->] (c) -- (d);
		 \draw[->] (2a) -- (2b);
		 \draw[->] (2b) -- (2c);
		 \draw[->] (2c) -- (2d);
		 \draw[->] (a) to node[left]{$\otimes_A B$} (2a);
		 \draw[->] (b) to node[left]{$\otimes_A B$ } (2b);
		 \draw[->] (c) to node[left]{$\otimes_A B$} (2c);
		 \end{tikzpicture}
    \end{center}
    Applicando il funtore $\Hom_A(\_,N)$ (che è esatto a sinistra) e indicando con $M_B$ $M\otimes B$:
    \begin{center}
      \begin{tikzpicture}[scale=3]
		 \node (a) at (-0.8,0) {$0$};
		 \node (b) at (0,0) {$\Hom_A(M,N)$};
		 \node (c) at (1,0) {$\Hom_A(A^n,N)$};
		 \node (d) at (2,0) {$\Hom_A(A^h,N)$};
		 \node (2a) at (-0.8,-0.3) {$0$};
		 \node (2b) at (0,-0.3) {$\Hom_A(M_B,N_B)$};
		 \node (2c) at (1,-0.3) {$\Hom_A(B^n,N_B)$};
		 \node (2d) at (2,-0.3) {$\Hom_A(B^h,N_B)$};
		 \draw[->] (a) to node[above]{$ $} (b);
		 \draw[->] (b) to node[above]{$ $}  (c);
		 \draw[->] (c) -- (d);
		 \draw[->] (2a) -- (2b);
		 \draw[->] (2b) -- (2c);
		 \draw[->] (2c) -- (2d);
% 		 \draw[->] (a) to node[left]{$\otimes_A B$} (2a);
% 		 \draw[->] (b) to node[left]{$\otimes_A B$ } (2b);
% 		 \draw[->] (c) to node[left]{$\otimes_A B$} (2c);
		 \end{tikzpicture}
    \end{center}
 Tuttavia applicando il funtore $B\otimes_A$ alla prima successione, usando il primo punto dell'osservazione 
 e sfruttando la piattezza si ha anche
  \begin{center}
      \begin{tikzpicture}[scale=4]
		 \node (2a) at (-0.5,-0.3) {$0$};
		 \node (2b) at (0,-0.3) {$B\otimes\Hom_A(M,N)$};
		 \node (2c) at (0.9,-0.3) {$B\otimes\Hom_A(A^n,N)$};
		 \node (2d) at (1.8,-0.3) {$B\otimes\Hom_A(A^h,N)$};
		 \draw[->] (2a) -- (2b);
		 \draw[->] (2b) -- (2c);
		 \draw[->] (2c) -- (2d);
		 \end{tikzpicture}
		  \end{center}
      Allora per il lemma dei cinque $\Hom_A(M_B,N_B)\simeq B\otimes\Hom_A(M,N)$.
 \end{itemize}
 \end{oss}
 Torniamo alla dimostrazione; per mostrare che $M$ è proiettivo dimostriamo che 
  è addendo diretto di un
  modulo libero, ossia che la successione
  \[
    A^n\rightarrow M \rightarrow 0
  \]
  spezza. 
 Applicando il funtore $\Hom_A(M,\_)$ abbiamo la successione
  \[
   \Hom_A(M,A^n)\rightarrow \Hom_A(M,M ) \rightarrow 0
 \]
 è esatta e proviamo che è esatta, in tal caso infatti trovando una controimmagine di $id_M$ avremmo la tesi.
 $\Hom_A(M,A^n)$ è lui stesso un $A$ modulo finitamente presentato, inoltre localizzando otteniamo
  \[
   (\Hom_A(M,A^n))_{\m}\rightarrow (\Hom_A(M,M ) )_{\m}\rightarrow 0
 \]
 Per quanto appena osservato (con $B=A_\m$) tale successione equivale  a 
  \[
   \Hom_{A_{\m}}(M_{\m},A^n_{\m})\rightarrow \Hom_{A_{\m}}(M_{\m},M_{\m} ) \rightarrow 0
 \]
 che però è esatta perché $M_\m$ è libero. Allora abbiamo concluso dato che essere suriettivo è una proprietà locale.
 \end{itemize}
 \end{proof}
 
 Consideriamo $A$ un dominio noetheriano e $M$ un modulo proiettivo e finitamente generato, allora se $p\in\Spec A$ si ha
 $M_p=A_p^n$ per qualche $n$ che non dipende da $p$ (dimostrare). Si definisce \emph{rango di M} tale $n$. In particolare se
 $\rank M=1$ diremo che è un \emph{fibrato lineare}.
 
 \begin{prop}
  Siano $M$ e $N$ due moduli proiettivi di rango rispettivamente $m$ e $n$. Allora
  \[
   \rank M\otimes N=mn
  \]
 \end{prop}

 \begin{de}
  Pic$(A)$ è l'insieme fibrati lineari a meno di isometria.
 \end{de}
 \begin{prop}
  Se $A$ è un dominio di Dedekind allora Pic$(A)$ col prodotto tensore è isomorfo al gruppo delle classi laterali.
 \end{prop}
%  \section{Fibrati vettoriali, idea geometrica}
%   Prendiamo due spazi topologici e $p\colon Y\rightarrow X$ e definiamo $Y\times_X Y\coloneqq\set{ (y_1,y_2)\ | \ p(y_1)=p(y_2) }$, definiamo 
%   poi due operazioni
%     \[
%      \oplus\colon Y\times_X Y \longrightarrow Y
%     \]
%     \[
%      \mathbb{K}\times Y \longrightarrow Y
%     \]
%     tali che rendano in un certo senso $Y\times_X Y$ uno spazio vettoriale. Inoltre chiediamo che per ogni $x\in X$ esiste
%     un intorno di $x$ $U\subseteq X$ tale che 
%     \[
%     \begin{matrix}
%      \inverti p (U) &\xrightarrow{\sim} & U\times \mathbb K^n
%     \end{matrix}
%     \]
%     dove la proiezione $\pi_U \colon U\times \mathbb K^n\rightarrow U$ è compatibile con le operazioni dello spazio.\\
%     Gli spazi di cui noi ci occupiamo sono $X=\Spec A$ e $Y=\Spec B$ con $A$ e $B$ $\mathbb{C}$-algebre; 
%     vista la particolare forma di questi spazi topologici in realtà possiamo considerare il morfismo
%     \[
%      \begin{matrix}
%      f & A &\longrightarrow & B
%     \end{matrix}
%     \]
%     l'operazione di somma allora diventa
%     \[
%      \oplus\colon\Spec (B\otimes_A B) \longrightarrow \Spec B
%     \]
%     Dove sto usando che $B\otimes_A B$ ha la proprietà del coprodotto fibrato; che tradotta a livello algebrico diventa
%     \[
%      \oplus\colon B \longrightarrow B\otimes_A B
%     \]
%     La moltiplicazione per scalare invece
%     \[
%      \mathbb{C}\times \Spec B \longrightarrow \Spec B
%     \]
%     diventa
%     \[
%      B\longrightarrow B[t]
%     \]
%     Inoltre vorremmo che localmente $\inverti p (U) \simeq U\times \mathbb C^n$. Ricordiamo che preso $a\in A$
%     $X_a=\Spec A_a$ è una base di aperti di $A$, allora è indotta da $f$ una mappa $\Spec B_a\rightarrow \Spec A_a$.
%     La condizione locale diventa allora  $\Spec B_a\simeq \Spec A_a\times \mathbb C^n\simeq \Spec (A_a[x_1,\dots,x_n])$.\\
%     $P=\set{b\in B\ |\ b=t\times b\in B[t]}$ \marginpar{????}    \\
%     Facciamo un esempio. Sia $B=A[x,y]\simeq A\times \mathbb{ C}^2$, con la $(a,u)+(a,w)=(a,u+w)$ e $\lambda (a,z)=(a,\lambda z)$.
%     \[
%      \begin{matrix}
%       A[x,y]&\longrightarrow & B[t]\\
% 	    x &\longmapsto & x\times t\\
% 	    y &\longmapsto & y\times t\\
% 	    a &\longmapsto & a\\
%      \end{matrix}
%     \]
%     $P=\set{b\in B\ |\ b=t\times b\in B[t]}=A_x\oplus A_y$ che è libero in $B$.
%     \begin{prop}
%      $P=\set{b\in B\ |\ b=t\times b\in B[t]}$ è un $A$ modulo e $P_a\simeq A_a^n$, cioè è localmente libero. 
%     \end{prop}
 \chapter{Funtori derivati e dimensione coomologica}
 \section{Dimensione coomologica}
 \begin{de}
  Sia $\A$ una categoria abeliana, 
   \[
    \dh(\A)\coloneqq \sup \set{n \mid \exists A,B\in\Ob\A \colon \Ext^n(A,B)\neq 0}
   \]
  è la \emph{dimensione coomologica} di $\A$.
 \end{de}
\begin{de}
  Sia $\A$ una categoria abeliana e $X\in\Ob\A$, 
   \[
    \dhp_\A(X)\coloneqq \sup \set{n \mid \exists B\in\Ob\A \colon \Ext^n(X,B)\neq 0}
   \]
  è la \emph{dimensione coomologica proiettiva} di $\A$; analogamente
  \[
    \dhi_\A(X)\coloneqq \sup \set{n \mid \exists A\in\Ob\A \colon \Ext^n(A,X)\neq 0}
  \]
 \end{de}
 
 \begin{oss}\label{oss:esattalunga}
  Abbiamo visto che a partire da un triangolo distinto
	    \begin{center}
	       \begin{tikzpicture}[scale=1.5]
		 \node (a) at (-1,0) {$\complx{A}$};
		 \node (b) at (0,0) {$\complx{B}$};
		 \node (c) at (1,0) {$\complx{C}$};
		 \node (d) at (2,0) {$\complx{A}[1]$};
		 \draw[->] (a) to node[above]{$f$} (b);
		 \draw[->] (b) to node[above]{$ $} (c);
		 \draw[->] (c) to node[above]{$ $} (d);
		\end{tikzpicture}
	    \end{center}
  e un complesso $\complx D$ abbiamo la successione esatta lunga 
  \begin{center}
  \begin{tikzpicture}[descr/.style={fill=white,inner sep=1.5pt}]
        \matrix (m) [
            matrix of math nodes,
            row sep=1em,
            column sep=2.5em,
            text height=1.5ex, text depth=0.25ex
        ]
        { \dots & \Hom(D,\complx A) & \Hom(\complx D,\complx B) & \Hom(\complx D,\complx C) \\
            & \Hom(\complx D,A[1]) & \Hom(\complx D,\complx B[1]) & \Hom(\complx D,\complx C[1]) \dots  \\
        };

        \path[overlay,->, font=\scriptsize,>=latex]
        (m-1-1) edge (m-1-2)
        (m-1-2) edge (m-1-3)
        (m-1-3) edge (m-1-4)
        (m-2-2) edge (m-2-3)
        (m-2-3) edge (m-2-4)
        (m-1-4) edge[out=355,in=175] (m-2-2);
\end{tikzpicture}
\end{center}

Scegliendo complessi della forma $\complx{\underline X}$ abbiamo allora
  \begin{center}
  \begin{tikzpicture}[descr/.style={fill=white,inner sep=1.5pt}]
        \matrix (m) [
            matrix of math nodes,
            row sep=1em,
            column sep=2.5em,
            text height=1.5ex, text depth=0.25ex
        ]
        { 0 & \Hom( D, A) & \Hom( D, B) & \Hom( D, C) \\
            & \Ext^1( D, A) & \Ext^1( D, B) & \Ext^1( D, C) \dots  \\
        };

        \path[overlay,->, font=\scriptsize,>=latex]
        (m-1-1) edge (m-1-2)
        (m-1-2) edge (m-1-3)
        (m-1-3) edge (m-1-4)
        (m-2-2) edge (m-2-3)
        (m-2-3) edge (m-2-4)
        (m-1-4) edge[out=355,in=175] (m-2-2);
\end{tikzpicture}
\end{center}
\end{oss}

\begin{oss}
 Se $P$ è proiettivo allora $\Ext^n(\complx {\underline P},\complx A)=0$ per ogni
 $n>0$ e quindi $\dhp(P)=0$. Analogamente se $I$ è iniettivo $\dhi(I)=0$.
\end{oss}

Gli oggetti proiettivi, abbiamo visto, giocano un ruolo centrale nel calcolo degli $\Ext$:
\begin{prop}
 Sia $\A$ una categoria abeliana con abbastanza proiettivi. Allora valgono i seguenti fatti:
 \begin{enumerate}
  \item Se per ogni $X\in \A$ e per ogni $i>0$ $\Ext^i(P,X)=0$, allora $P$ è proiettivo.
  \item Siano $A,B\in\A$. Se esiste una successione esatta 
  \begin{center}
   \begin{tikzpicture}
    \node (a) at (1,0) {$0$};
    \node (b) at (2,0) {$B$};
    \node (c) at (3,0) {$P^{k}$};
    \node (d) at (4,0) {$\dots$};
    \node (e) at (5,0) {$P^1$};
    \node (f) at (6,0) {$A$};
    \node (g) at (7,0) {$0$};
    \draw[->] (a) -- (b);
    \draw[->] (b) -- (c);
    \draw[->] (c) -- (d);
    \draw[->] (d) -- (e);
    \draw[->] (e) -- (f);
    \draw[->] (f) -- (g);
   \end{tikzpicture}
  \end{center}
   con $P^i$ proiettivi, allora
   \[
    \dhp B=\dhp A -k
   \]
 \end{enumerate}
\end{prop}

\begin{proof}
 \begin{enumerate}
  \item Dato un morfismo suriettivo $\beta\colon A\rightarrow B$, indicando con $\alpha=\ker\beta$ abbiamo una successione esatta
  corta
	    \begin{center}
	       \begin{tikzpicture}[scale=1.5]
		 \node (a) at (-1,0) {$0$};
		 \node (b) at (0,0) {$K$};
		 \node (c) at (1,0) {$A$};
		 \node (d) at (2,0) {$B$};
		 \node (e) at (3,0) {$0$};
		 \draw[->] (a) to node[above]{$ $} (b);
		 \draw[->] (b) to node[above]{$\alpha$} (c);
		 \draw[->] (c) to node[above]{$\beta $} (d);
		 \draw[->] (d) to node[above]{$ $} (e);
		\end{tikzpicture}
	    \end{center}
    Per l'osservazione \ref{oss:esattalunga} e usando che  $\Ext^i(P,A)=0$ per ipotesi abbiamo che 
    \begin{center}
	\begin{tikzpicture}[scale=2.3]
	 \node (a) at (-0.8,0) {$0$};
	 \node (b) at (0,0) {$\Hom(P,K)$};
	 \node (c) at (1,0) {$\Hom(P,A)$};
	 \node (d) at (2,0) {$\Hom(P,B)$};
	 \node (e) at (2.8,0) {$0$};
	 \draw[->] (a) -- (b);
	 \draw[->] (b) -- (c);
	 \draw[->] (c) -- (d);
	 \draw[->] (d) -- (e);
   \end{tikzpicture}
  \end{center} 
  è esatta; dunque esiste per ogni $f\in \Hom(P,B)$ esiste $g\in \Hom(P,A)$ tale che $\beta\circ g=f$. Questo è equivalente a dire 
  $P$ è proiettivo.
  \item Mostriamo la tesi per induzione su $k$. \\
	Se $k=0$ Abbiamo che $0\rightarrow B\rightarrow A\rightarrow 0$ è esatta, ossia $A$ $B$ sono isomorfi. \\
	Se $k=1$ Allora $0\rightarrow B\rightarrow P \rightarrow A\rightarrow 0$ è esatta. Passando alla successione esatta
	lunga degli $\Hom(\_,D)$ otteniamo
    \begin{center} 	
    \begin{tikzpicture}[descr/.style={fill=white,inner sep=1.5pt}]
        \matrix (m) [
            matrix of math nodes,
            row sep=1em,
            column sep=2.5em,
            text height=1.5ex, text depth=0.25ex
        ]
        { 0 & \Hom( A, D) & \Hom( P, D) & \Hom( B, D) \\
            & \Ext^1(A, D) & \Ext^1( P, D) & \Ext^1( B, D) \dots  \\
            & \Ext^i( A	, D) & \Ext^i( P, D) & \Ext^i( B, D) \dots  \\
        };

        \path[overlay,->, font=\scriptsize,>=latex]
        (m-1-1) edge (m-1-2)
        (m-1-2) edge (m-1-3)
        (m-1-3) edge (m-1-4)
        (m-2-2) edge (m-2-3)
        (m-2-3) edge (m-2-4)
        (m-1-4) edge[out=355,in=175] (m-2-2)
        (m-2-4) edge[out=355,in=175] (m-3-2)
        (m-3-2) edge (m-3-3)
        (m-3-3) edge (m-3-4);
    \end{tikzpicture}
    \end{center}
	Dato che $P$ è proiettivo $\Ext^i(P,D)=0$ per ogni $i>0$, quindi $\Ext^i(A,D)\simeq\Ext^{i+1}(B,D)$.
	Passando al minimo si ottiene $\dhp B=\dhp A +1$,\\
	Se $k>1$ Dalla successione esatta otteniamo due altre successioni esatte 
	\begin{center}
	\begin{tikzpicture}[scale=1.2]
	\node (a) at (1,0) {$0$};
	\node (b) at (2,0) {$B$};
	\node (c) at (3,0) {$P^{k}$};
	\node (d) at (4.3,0) {$P^{k-1}$};
	\node (e) at (5.3,0) {$\dots$};
	\node (f) at (6.3,0) {$P^1$};
	\node (g) at (7.3,0) {$A$};
	\node (h) at (8,0) {$0$};
	\node (C) at (3.6,-0.8) {$C$};
	\node (A) at (3,-1.6) {$0$};
	\node (B) at (4.3,-1.6) {$0$};
	\draw[->] (a) -- (b);
	\draw[->] (b) -- (c);
	\draw[->] (c) -- (d);
	\draw[->] (d) -- (e);
	\draw[->] (e) -- (f);
	\draw[->] (f) -- (g);
	\draw[->] (g) -- (h);
	\draw[->] (f) -- (g);
	\draw[->] (c) -- (C);
	\draw[->] (C) -- (d);
	\draw[->] (A) -- (C);
	\draw[->] (C) -- (B);
      \end{tikzpicture}
      \end{center}
      Allora per ipotesi induttiva $\dhp A=\dhp C-k+1$ e per il caso $k=1$ invece $\dhp C=\dhp B - 1$, da cui
      \[
       \dhp A=\dhp B - 1-k+1=\dhp B-k.
      \]

 \end{enumerate}
\end{proof}
  
  Questa proposizione ci permette anche di dare una condizione equivalente alla caratterizzazione della dimensione proiettiva
  in termini di risoluzioni proiettive minimali.
 \begin{cor}
  Sia $\A$ una categoria con abbastanza proiettivi,
  $\dhp M=k$ se e solo se esiste
  \[
    0\rightarrow P^{-k} \rightarrow \dots \rightarrow P^{0} \rightarrow M\rightarrow 0
  \]
  esatta di lunghezza minima e con $P^i$ oggetti proiettivi.
 \end{cor}

 \begin{proof}
  Questa condizione è ovviamente sufficiente. Viceversa, per ipotesi esiste 
  \[
    0\rightarrow N\rightarrow P^{-k+1} \rightarrow \dots \rightarrow P^{0} \rightarrow M\rightarrow 0
  \]
  Ma  $\dhp (N)=\dhp (M) -k=0$, dunque $N$ è proiettivo e questa è la successione cercata.
 \end{proof}

 \section{Funtori Derivati}
 Il funtore $F\coloneqq\Hom(M,\_)$ con  $M\in \Ob \A$ è esatto a sinistra. Abbiamo visto che però presa
 una sequenza esatta corta
 \begin{center}
 \begin{tikzpicture}[scale=1]
	 \node (a) at (-1,0) {$0$};
 	 \node (b) at (0,0) {$A$};
	 \node (c) at (1,0) {$B$};
 	 \node (d) at (2,0) {$C$};
 	 \node (a3) at (3,0) {$0$};
	 \draw[->] (a) -- (b);
	 \draw[->] (b) to node[above]{$ $} (c);
	 \draw[->] (c) to node[above]{$ $} (d);
	 \draw[->] (d) -- (a3);
       \end{tikzpicture}
      \end{center} 
      applicando $F$ è ben definita la sequenza esatta lunga
          \begin{center} 	
    \begin{tikzpicture}[descr/.style={fill=white,inner sep=1.5pt}]
        \matrix (m) [
            matrix of math nodes,
            row sep=1em,
            column sep=2.5em,
            text height=1.5ex, text depth=0.25ex
        ]
        { 0 & \Hom( M,A) & \Hom( M,B) & \Hom(M,C) \\
            & \Ext^1(M,A) & \Ext^1( M,B) & \Ext^1( M, C) \dots  \\
            & \Ext^i( M,A ) & \Ext^i( M, B) & \Ext^i( M, C) \dots  \\
        };

        \path[overlay,->, font=\scriptsize,>=latex]
        (m-1-1) edge (m-1-2)
        (m-1-2) edge (m-1-3)
        (m-1-3) edge (m-1-4)
        (m-2-2) edge (m-2-3)
        (m-2-3) edge (m-2-4)
        (m-1-4) edge[out=355,in=175] (m-2-2)
        (m-2-4) edge[out=355,in=175] (m-3-2)
        (m-3-2) edge (m-3-3)
        (m-3-3) edge (m-3-4);
    \end{tikzpicture}
    \end{center} 
  che in un certo senso misura quanto $F$ non sia esatto a sinistra. Vorremmo adesso estendere questo concetto ad un 
  funtore qualsiasi che sia parzialmente esatto.\\
  
   Sia $\A$ una categoria abeliana con abbastanza iniettivi e proiettivi e $\B$ anch'essa abeliana. Preso un funtore 
   \[
    F\colon \A\longrightarrow \B
   \]
   esatto a sinistra, definiamo \emph{funtore derivato destro}
   \[
    RF\colon \der\A^+\longrightarrow \der\B^+
   \]
   un funtore tale che
   \begin{itemize}
    \item mandi triangoli distinti in triangoli distinti,
    \item sia \textquotedblleft collegabile\textquotedblright \ con $F$.
   \end{itemize}
   In particolare se $A\in\Ob\A$ allora $H^0(RF(\complx{\underline A})=FA$.\\

 Una prima possibilità per definirlo è 
   \[
      \bar F\colon \Com\A^+\longrightarrow \Com\B^+
   \]
   In modo che $\complx{A}\mapsto \complx{ FA}$; tuttavia passando già in $\Kom$ $\bar F$ non manda triangoli 
   distinti in triangoli distinti.\\
   Analogamente potremmo pensare di definire
   \[
      K^+ F\colon \Kom\A^+\longrightarrow \Kom\B^+
   \]
   In modo che $\complx{A}\mapsto \complx{ FA}$; questo manda triangoli distinti in triangoli distinti, infatti consideriamo
	\begin{center}
	       \begin{tikzpicture}[scale=2]
		 \node (a) at (-0.7,0) {$\complx{X}$};
		 \node (b) at (0,0) {$\complx{Y}$};
		 \node (c) at (1,0) {$\Cono (f)$};
		 \node (d) at (2,0) {$\complx{X}[1]$};
		 \draw[->] (a) to node[above]{$f$} (b);
		 \draw[->] (b) -- (c);
		 \draw[->] (c) to node[above]{$-\delta$} (d);
		\end{tikzpicture}
	    \end{center}
    in $\Kom\A^+$. Applicando $F$ si ha
	  \begin{center}
	       \begin{tikzpicture}[scale=2]
		 \node (a) at (-1,0) {$\complx{FX}$};
		 \node (b) at (0,0) {$\complx{FY}$};
		 \node (c) at (1,0) {$F(\Cono (f))$};
% 		 \node (d) at (2,0) {$\complx{X}[1]$};
		 \draw[->] (a) to node[above]{$Ff$} (b);
		 \draw[->] (b) -- (c);
% 		 \draw[->] (c) to node[above]{$-\delta$} (d);
		\end{tikzpicture}
	    \end{center}
	 Questo però da un triangolo distinto perché
	 \[
	  F(\Cono (f))^n=FA^n\oplus FB^n
	 \]
	 e il bordo è 
	 \[
	  \begin{pmatrix}
	   -F\partial_A & 0\\
	   Ff  & F\partial_b
	  \end{pmatrix}
	 \]
	 perciò  $F(\Cono (f))=\Cono(Ff)$.\\
	 Ancora una volta però questo funtore non è quello giusto perché non si comporta bene rispetto ai quasi
	 isomorfismi. Preso $f$ un quasi isomorfismo, allora passando alla categoria derivata otteniamo un isomorfismo. 
	 Se lo completiamo a triangolo distinto in $\Kom$
	 \begin{center}
	       \begin{tikzpicture}[scale=1.5]
		 \node (a) at (-1,0) {$\complx{A}$};
		 \node (b) at (0,0) {$\complx{B}$};
		 \node (c) at (1,0) {$\complx{C}$};
		 \node (d) at (2,0) {$\complx{A}[1]$};
		 \draw[->] (a) to node[above]{$f$} (b);
		 \draw[->] (b) to node[above]{$ $} (c);
		 \draw[->] (c) to node[above]{$ $} (d);
		\end{tikzpicture}
	    \end{center}
	 e applichiamo $KF^+$ otteniamo
	 \begin{center}
	       \begin{tikzpicture}[scale=2]
		 \node (a) at (-1,0) {$F\complx{A}$};
		 \node (b) at (0,0) {$F\complx{B}$};
		 \node (c) at (1,0) {$F\complx{C}$};
		 \node (d) at (2,0) {$F\complx{A}[1]$};
		 \draw[->] (a) to node[above]{$Ff$} (b);
		 \draw[->] (b) to node[above]{$ $} (c);
		 \draw[->] (c) to node[above]{$ $} (d);
		\end{tikzpicture}
	    \end{center}
	$f$ è un quasi isomorfismo perciò $\complx{H}(C)=0$, allora dovremmo avere $\complx{H}(FC)=0$ ma ciò non accade.
	Infatti $F$ è esatto solo a sinistra e quindi applicandolo a
	\[
	0\rightarrow C^0\rightarrow C^1\rightarrow C^2\rightarrow C^3\rightarrow \dots
	\]
	si perde l'esattezza già in $FC^1$.\\
	
	Questo problema è centrale per la buona definizione del funtore derivato a destra,  quindi un Lemma che ci aiuti ad 
	aggirare questo ostacolo.
	\begin{lemma}
	 Sia $\complx{A}\in\Com^+(\A)$ tale che $A^i$ è un oggetto iniettivo per ogni $i$ e \emph{aciclico}, ossia $H^i(A)=0$ per ogni $i$.
	 Allora $\complx{A}$ spezza
	    \[
	     A^i\simeq B^i(A)\oplus B^{i+1}(A)
	    \]
	\end{lemma}
	\begin{proof}
	 Poichè $A^0$ è iniettivo allora 
	 \[
	 0\rightarrow A^0\rightarrow A^1
	 \]
	allora spezza, quindi $A^1=A^0\oplus C^1=B^1\oplus C^1$ e poichè gli $A^i$ sono 
	iniettivi allora lo è anche $C^1$.\\
	Il complesso è aciclico, perciò $\ker\partial^1_A=B^1$ e dunque la restrizione a $C^1$ è iniettiva:
	 \[
	 0\rightarrow C^1\rightarrow A^1
	 \]
	 ma allora $C^1\simeq B^2$ e dunque $A^1\simeq  B^1(A)\oplus B^{2}(A)$.
	 Iterando questo procedimento si ha la tesi.
	\end{proof}
	
	\begin{oss}
	 Se $\complx{A}\in\Com^+(\A)$ tale che $A^i$ è un oggetto iniettivo per ogni $i$ e aciclico (come nel Lemma)
	 Applicando un funtore $F$ esatto a sinistra otteniamo $FA^i\simeq FB^i\oplus FB^{i+1}$
	 e il complesso 
	 \[
	  0\rightarrow FA^0\rightarrow FA^1\rightarrow FA^2\rightarrow FA^3\rightarrow \dots
	 \]
	in cui il bordo agisce come l'identità su $FB^{i+1}$ e zero su $FB^{i}$, anch'esso risulta aciclico.
	\end{oss}
	
	 L'idea per definire il complesso derivato è quindi quella di sfruttare che su questi particolari complessi 
	 applicando un funtore esatto a sinistra viene rispettata la comologia. Ricordiamo inoltre
	 che se una categoria $\A$ ha abbastanza iniettivi, preso un complesso $\complx{A}\in\Com^+\A$ esiste un complesso
	 $\complx{I}_A\in\Com^+\A$ fatto di oggetti iniettivi e un quasi isomorfismo 
	 \[
	  i_A\colon\complx{A}\longmapsto\complx{I}_A
	 \]
	  tale che $i_A^n$ è un monomorfismo per ogni $n$.\\
	  Abbiamo anche osservato grazie al proprietà legate all'iniettività
	  $$\Hom_{\der\A^+}(\complx{I_A},\complx{I_B})=\Hom_{\Kom^+\A}(\complx{I_A},\complx{I_B}).$$
	  \noindent
	  Questo fatto unito a quanto
	  sopra ci da che esiste il seguente isomorfismo:
	  \[
	  \begin{matrix}
	  \Hom_{\der\A^+}(\complx{A},\complx{B})& \xrightarrow {\sim} &\Hom_{\Kom^+\A}(\complx{I_A},\complx{I_B})\\
	  f&\longmapsto& i_B\circ f\circ \inverti{i_A}
	  \end{matrix}
	  \]
	  Facilmente, grazie a queste osservazioni, otteniamo il seguente fatto
	  \begin{prop}\label{prop:eqiniett}
	   Sia  $\A$ una categoria con abbastanza iniettivi. La categoria 
	   dei complessi iniettivi limitati inferiormente $\Kom(\mathcal{I}_{\A})^+$ è naturalmente 
	   equivalente a $\der{\A}^+$.
	  \end{prop}
	  Questa proposizione da che la seguente è una buona definizione:
	\begin{de}
	 Sia  $\A$ una categoria con abbastanza iniettivi e $F$ un funtore esatto a sinistra. Il
	 funtore derivato a destra
	\[\begin{matrix}
	   RF\colon &\der\A^+&\longrightarrow &\der\B^+\\
		    &\complx{A}&\longmapsto& F(\complx{I}_A)\\
		    &f&\mapsto & RF(g)
	  \end{matrix}
	\]
	dove se $f\in\Hom_{\der\A}(\complx{A},\complx{B})$ allora
	$g=i_B\circ f\circ \inverti{i_A}$.
	\end{de}
	Analogamente si dimostra che
	\begin{prop}
	   Sia  $\A$ una categoria con abbastanza proiettivi. La categoria 
	   dei complessi iniettivi limitati superiormente $\Kom(\mathcal{P}_{\A})^-$ è naturalmente 
	   equivalente a $\der{\A}^-$.
	  \end{prop}
	 Detto poi $p_A\colon\complx{P}_A\rightarrow\complx{A}$ un quasi isomorfismo tra una risoluzione proiettiva e un complesso, 
	 abbiamo la buona definizione del \emph{funtore derivato a sinistra}
	 \begin{de}
	 Sia  $\A$ una categoria con abbastanza proiettivi e $F$ un funtore esatto a destra. Il
	 funtore derivato a sinistra è
	\[\begin{matrix}
	   LF\colon &\der\A^-&\longrightarrow &\der\B^-\\
		    &\complx{A}&\longmapsto& F(\complx{P}_A)\\
		    &f&\mapsto & LF(h)
	  \end{matrix}
	\]
	dove se $f\in\Hom_{\der\A}(\complx{A},\complx{B})$ allora
	$h=\inverti{p_B}\circ f\circ {p_A}$.
	\end{de}
	
	\begin{de}
	 Consideriamo la categoria degli $A$ moduli. Il funtore $F(N)=M\otimes_{A}N$ è esatto a destra.
	 Definiamo 
	 \[
	  \Tor^i_{A}(M,N)\coloneqq H^{-i}(LF(N))
	 \]
	\end{de}
	\begin{esem}
	 Siano $M=\faktor{\Z}{m}$ e $N=\faktor{\Z}{n}$ due $\Z$ moduli.
	 Consideriamo risoluzione proiettiva (libera) di 
	 $N$
	 \begin{center}
       \begin{tikzpicture}[scale=1.8]
		 \node (a0) at (-3,0) {$\ $};
		 \node (a1) at (-2,0) {$0 $};
		 \node (a) at (-1,0) {$\Z$};
		 \node (b) at (0,0) {$\Z$};
		 \node (c) at (1,0) {$0$};
		 \node (d) at (2,0) {$0$};
		 \node (a3) at (3,0) {$\ $};
		 \node (2a0) at (-3,-1) {$\ $};
		 \node (2a1) at (-2,-1) {$0 $};
		 \node (2a) at (-1,-1) {$0$};
		 \node (2b) at (0,-1) {$\faktor{\Z}{n\Z}$};
		 \node (2c) at (1,-1) {$0$};
		 \node (2d) at (2,-1) {$0$};
		 \node (2a3) at (3,-1) {$\ $};
		 \draw[->] (a) to node[above]{$ $} (b);
		 \draw[->] (b) to node[above]{$ $}  (c);
		 \draw[->] (c) -- (d);
		 \draw[->] (a1) -- (a);
		 \draw[->,dashed] (a0) -- (a1);
		 \draw[->,dashed] (d) -- (a3);	
		 \draw[->] (2a) -- (2b);
		 \draw[->] (2b) -- (2c);
		 \draw[->] (2c) -- (2d);
		 \draw[->] (2a1) -- (2a);
		 \draw[->,dashed] (2a0) -- (2a1);
		 \draw[->,dashed] (2d) -- (2a3);
		 \draw[->] (a1) to node[left]{$0$} (2a1);
		 \draw[->] (a) to node[left]{$0$} (2a);
		 \draw[->] (b) to node[left]{$\pi$} (2b);
		 \draw[->] (c) to node[left]{$0$} (2c);
		 \draw[->] (d) to node[left]{$0$} (2d);
		 \end{tikzpicture}
    \end{center}
	 Se applichiamo $M\otimes\_$  otteniamo
	 \begin{center}
	 \begin{tikzpicture}[scale=2]
	 \node (a) at (-1,0) {$0$};
	 \node (b) at (0,0) {$M\otimes\Z$};
	 \node (c) at (1,0) {$M\otimes\Z$};
	 \node (d) at (2,0) {$0$};
	 \draw[->] (a) -- (b);
	 \draw[->] (b) to node[above]{$n$} (c);
	 \draw[->] (c) -- (d);
	 \end{tikzpicture}
	 \end{center}
	 e ricordando che $M\otimes\Z=M$ 
	 \begin{center}
	 \begin{tikzpicture}[scale=2]
	 \node (a) at (-1,0) {$0$};
	 \node (b) at (0,0) {$\faktor{\Z}{m\Z}$};
	 \node (c) at (1,0) {$\faktor{\Z}{m\Z}$};
	 \node (d) at (2,0) {$0$};
	 \draw[->] (a) -- (b);
	 \draw[->] (b) to node[above]{$n$} (c);
	 \draw[->] (c) -- (d);
	 \end{tikzpicture}
	 \end{center}
	 Se calcoliamo perciò la coomologia vediamo che
	 \[
	  \Tor^0(M,N)=\faktor{\Z}{m\Z}\otimes \faktor{\Z}{n\Z}=\faktor{\Z}{d\Z}
	 \]
	  \[
	  \Tor^1(M,N)=\faktor{\Z}{d\Z}
	 \]
	 con $d=\gcd(m,n)$.
	\end{esem}
	Mostriamo che i funtori derivati così definiti hanno le proprietà che avevamo richiesto all'inizio del capitolo.
	\begin{lemma}\*
	 \begin{enumerate}[i.]
	  \item Se $A\in\Ob\A$ allora $H^0(RF(\complx{\underline A}))=FA$.
	  \item $RF$ porta triangoli distinti in triangoli distinti.
	 \end{enumerate}
	\end{lemma}
      \begin{proof}
       i. A partire da una risoluzione iniettiva di $\complx{\underline A}$	 
       \begin{center}
       \begin{tikzpicture}[scale=1.8]
		 \node (a0) at (-3,0) {$\ $};
		 \node (a1) at (-2,0) {$0$};
		 \node (a) at (-1,0) {$A$};
		 \node (b) at (0,0) {$0$};
		 \node (c) at (1,0) {$0$};
		 \node (d) at (2,0) {$0$};
		 \node (a3) at (3,0) {$\ $};
		 \node (2a0) at (-3,-1) {$\ $};
		 \node (2a1) at (-2,-1) {$0 $};
		 \node (2a) at (-1,-1) {$I^0$};
		 \node (2b) at (0,-1) {$I^1$};
		 \node (2c) at (1,-1) {$I^2$};
		 \node (2d) at (2,-1) {$I^3$};
		 \node (2a3) at (3,-1) {$\ $};
		 \draw[->] (a) to node[above]{$ $} (b);
		 \draw[->] (b) to node[above]{$ $}  (c);
		 \draw[->] (c) -- (d);
		 \draw[->] (a1) -- (a);
		 \draw[->,dashed] (a0) -- (a1);
		 \draw[->,dashed] (d) -- (a3);	
		 \draw[->] (2a) -- (2b);
		 \draw[->] (2b) -- (2c);
		 \draw[->] (2c) -- (2d);
		 \draw[->] (2a1) -- (2a);
		 \draw[->,dashed] (2a0) -- (2a1);
		 \draw[->,dashed] (2d) -- (2a3);
		 \draw[->] (a1) to node[left]{$0$} (2a1);
		 \draw[->] (a) to node[left]{$j$} (2a);
		 \draw[->] (b) to node[left]{$0$} (2b);
		 \draw[->] (c) to node[left]{$0$} (2c);
		 \draw[->] (d) to node[left]{$0$} (2d);
		 \end{tikzpicture}
      \end{center}
      otteniamo una successione esatta
      \begin{center}
	 \begin{tikzpicture}[scale=1.5]
	 \node (a) at (-1,0) {$0$};
	 \node (b) at (0,0) {$A$};
	 \node (c) at (1,0) {$I^0$};
	 \node (d) at (2,0) {$I^1$};
	 \draw[->] (a) -- (b);
	 \draw[->] (b) to node[above]{$ $} (c);
	 \draw[->] (c) -- (d);
	 \end{tikzpicture}
	 \end{center}
      Poichè $F$ è esatto a sinistra
      \begin{center}
	 \begin{tikzpicture}[scale=1.5]
	 \node (a) at (-1,0) {$0$};
	 \node (b) at (0,0) {$FA$};
	 \node (c) at (1,0) {$FI^0$};
	 \node (d) at (2,0) {$FI^1$};
	 \draw[->] (a) -- (b);
	 \draw[->] (b) to node[above]{$ $} (c);
	 \draw[->] (c) -- (d);
	 \end{tikzpicture}
	 \end{center}
	 rimane esatta. Dato che $RF(\complx{\underline A})=F(\complx I)$
	 allora si ha proprio $H^0(RF(\complx{\underline A}))=FA$.\\
	 ii. Ogni triangolo 
	 \begin{center}
	       \begin{tikzpicture}[scale=1.5]
		 \node (a) at (-1,0) {$\complx{A}$};
		 \node (b) at (0,0) {$\complx{B}$};
		 \node (c) at (1,0) {$\complx{C}$};
		 \node (d) at (2,0) {$\complx{A}[1]$};
		 \draw[->] (a) to node[above]{$ $} (b);
		 \draw[->] (b) to node[above]{$ $} (c);
		 \draw[->] (c) to node[above]{$ $} (d);
		\end{tikzpicture}
	    \end{center}
      grazie alla Proposizione \ref{prop:eqiniett} è quasi isomorfo al triangolo
      \begin{center}
	       \begin{tikzpicture}[scale=1.8]
		 \node (a) at (-1,0) {$\complx{I}_A$};
		 \node (b) at (0,0) {$\complx{I}_B$};
		 \node (c) at (1,0) {$\complx{I}_C$};
		 \node (d) at (2,0) {$\complx{I}_A[1]$};
		 \draw[->] (a) to node[above]{$ $} (b);
		 \draw[->] (b) to node[above]{$ $} (c);
		 \draw[->] (c) to node[above]{$ $} (d);
		\end{tikzpicture}
      \end{center}
      Applicare $RF$ al primo è quindi come applicarlo al secondo, ci possiamo quindi ridurre a lavorare in $\Kom^+$.
      Per concludere basta ripetere le osservazioni fatte per $K^+F$, il nostro triangolo è isomorfo a 
      \begin{center}
	       \begin{tikzpicture}[scale=2]
		 \node (a) at (-1,0) {$\complx{I}_A$};
		 \node (b) at (0,0) {$\complx{I}_B$};
		 \node (c) at (1,0) {$\Cono(g)$};
		 \node (d) at (2,0) {$\complx{I}_A[1]$};
		 \draw[->] (a) to node[above]{$g$} (b);
		 \draw[->] (b) to node[above]{$ $} (c);
		 \draw[->] (c) to node[above]{$ $} (d);
		\end{tikzpicture}
      \end{center}
      applicando il funtore $F$ e ricordando che $F(\Cono(g))=\Cono(Fg)$ si ha
	      \begin{center}
	       \begin{tikzpicture}[scale=2]
		 \node (a) at (-1,0) {$F\complx{I}_A$};
		 \node (b) at (0,0) {$F\complx{I}_B$};
		 \node (c) at (1,0) {$\Cono(Fg)$};
		 \node (d) at (2.3,0) {$F(\complx{I}_A)[1]$};
		 \draw[->] (a) to node[above]{$Fg$} (b);
		 \draw[->] (b) to node[above]{$ $} (c);
		 \draw[->] (c) to node[above]{$ $} (d);
		\end{tikzpicture}
      \end{center}
      che è un triangolo distinto.
      \end{proof}
      
      \begin{de}
       Sia $A\in\Ob\A$. Si definisce
       \[
        R^iF(A)\coloneqq H^i(RF(A))
       \]
	inoltre diremo che $A$ è \emph{adatto ad F} se $R^iF(A)=0$ per ogni $i$.
      \end{de}
      \begin{esem}\
       \begin{itemize}
        \item Gli oggetti adatti  per il tensore sono tutti e soli i moduli piatti.
        \item Gli oggetti adatti per $\Hom_A(M,\_)$ sono tutti e soli i moduli proiettivi.
       \end{itemize}
      \end{esem}

      \begin{lemma}\label{lemma:adatti_aciclici}
      Consideriamo una successione esatta
      \begin{center}
	\begin{tikzpicture}[scale=1.5]
	 \node (a) at (-1,0) {$0$};
	 \node (b) at (0,0) {$A$};
	 \node (c) at (1,0) {$B$};
	 \node (d) at (2,0) {$C$};
	 \node (a3) at (3,0) {$0$};
	 \draw[->] (a) -- (b);
	 \draw[->] (b) to node[above]{$f$} (c);
	 \draw[->] (c) to node[above]{$g$} (d);
	 \draw[->] (d) -- (a3);
   \end{tikzpicture}
  \end{center}
       \begin{enumerate}
        \item Se A è adatto allora 
        \begin{center}
	\begin{tikzpicture}[scale=1.5]
	 \node (a) at (-1,0) {$0$};
	 \node (b) at (0,0) {$FA$};
	 \node (c) at (1,0) {$FB$};
	 \node (d) at (2,0) {$FC$};
	 \node (a3) at (3,0) {$0$};
	 \draw[->] (a) -- (b);
	 \draw[->] (b) to node[above]{$Ff$} (c);
	 \draw[->] (c) to node[above]{$Fg$} (d);
	 \draw[->] (d) -- (a3);
	 \end{tikzpicture}
	 \end{center}
	  è esatta;
	\item Se $A,B$ adatti allora $C$ adatto. Se $A,C$ adatti allora $B$ adatto.
	\item Se $\complx A\in\Kom^+\A$ è aciclico e gli $A^i$ sono tutti adatti, allora $F\complx{A}$ è aciclico.
	\item Gli oggetti iniettivi sono adatti.
       \end{enumerate}
      \end{lemma}
      \begin{proof}
       Dall'esattezza, abbiamo in $\Kom^+\A$ i quasi isomorfismi
       \begin{center}
       \begin{tikzpicture}[scale=1.8]
		 \node (a) at (-1,0) {$\complx{\underline A}$};
		 \node (b) at (0,0) {$\complx{\underline B}$};
		 \node (c) at (1,0) {$\Cono(f)$};
		 \node (d) at (2,0) {$\complx{\underline A}[1]$};
		 \node (1a) at (-1,1) {$\complx I_A$};
		 \node (1b) at (0,1) {$\complx I_B$};
		 \node (1c) at (1,1) {$\complx I_C$};
		 \node (1d) at (2,1) {$ $};
		 \node (2a) at (-1,-1) {$\complx{\underline A}$};
		 \node (2b) at (0,-1) {$\complx{\underline B}$};
		 \node (2c) at (1,-1) {$\complx{\underline C}$};
		 \node (2d) at (2,-1) {$ $};
		 \draw[->] (a) to node[above]{$ $} (b);
		 \draw[->] (b) to node[above]{$ $}  (c);
		 \draw[->] (c) -- (d);
		 \draw[->] (1a) to node[above]{$ $} (1b);
		 \draw[->] (1b) to node[above]{$ $}  (1c);
		 \draw[->] (2a) -- (2b);
		 \draw[->] (2b) -- (2c);
		 \draw[->] (a) -- (2a);
		 \draw[->] (b) -- (2b);
		 \draw[->] (c) -- (2c);
		 \draw[->] (1a) -- (a);
		 \draw[->] (1b) -- (b);
		 \draw[->] (1c) -- (c);
		 \end{tikzpicture}
      \end{center}
      È ben definita la successione esatta lunga
      \[
	0\rightarrow FA \rightarrow FB \rightarrow FC \rightarrow R^1FA\rightarrow R^1FB\rightarrow \dots
      \]
      infatti il triangolo 
      \begin{center}
	       \begin{tikzpicture}[scale=1.5]
		 \node (a) at (-1,0) {$\complx{I}_A$};
		 \node (b) at (0,0) {$\complx{I}_B$};
		 \node (c) at (1,0) {$\complx{I}_C$};
		 \node (d) at (2,0) {$\complx{I}_A[1]$};
		 \draw[->] (a) to node[above]{$ $} (b);
		 \draw[->] (b) to node[above]{$ $} (c);
		 \draw[->] (c) to node[above]{$ $} (d);
		\end{tikzpicture}
      \end{center}
      è distinto e applicatogli $F$ va in un triangolo distinto, passando perciò in comologia 
      e ricordando che le frecce nel diagramma all'inizio sono quasi isomorfismi (isomorfismi
      in coomologia), si ha la tesi.\\
      Alla luce di questo fatto 1. e 2. sono ovvi.\\
      3. Consideriamo il complesso 
      \[
	0\rightarrow A^0 \rightarrow A^1 \rightarrow A^2 \rightarrow A^3 \rightarrow \dots
      \]
      dato che è aciclico abbiamo le successioni esatte\
      \[
	0\rightarrow A^0 \rightarrow A^1 \rightarrow B^2 \rightarrow 0
      \]
      e
      \[
	0\rightarrow B^2 \rightarrow A^2 \rightarrow B^3 \rightarrow 0
      \]
      Usando 2., otteniamo che $B^2$ è adatto e allora lo è anche $B^3$. \\
      Applichiamo $F$: 
      \[
	0\rightarrow FA^0 \rightarrow FA^1 \rightarrow FA^2 \rightarrow FA^3 \rightarrow \dots
      \]
      Per 1. abbiamo che 
      \[
	0\rightarrow FA^0 \rightarrow FA^1 \rightarrow FB^2 \rightarrow 0
      \]
      è esatta.  
      Per mostrare che $F(\complx A)$ è aciclico dobbiamo innanzi tutto far vedere che
      \[
       \Imm F\partial_A^1=\ker F\partial_A^2
      \]
      Consideriamo
      \[
	0\rightarrow FB^2 \xrightarrow{\alpha} FA^2 \xrightarrow{\beta} FB^3 \rightarrow 0
      \]
      è esatta visto che è $B^2$ è adatto, allora
      \[
       \Imm F\partial_A^1=\Imm{\alpha}=\ker{\beta}=\ker F\partial_A^2
      \]
      Iterando questo procedimento si ha la tesi.
      \end{proof}
      
      \begin{teo}
       Se $\complx{A}\in\Com^+$ è di oggetti adatti, allora
       \[
        RF(\complx{A})=F(\complx{A})
       \]
      \end{teo}
      \begin{proof}
       Il quasi isomorfismo 
	 \[
	  i_A\colon\complx{A}\longrightarrow\complx{I}_A
	 \]
      in $\Kom^+$ si completa a triangolo distinto. 
      \begin{center}
	       \begin{tikzpicture}[scale=1.3]
		 \node (a) at (-1,0) {$\complx A$};
		 \node (b) at (0,0) {$\complx{I_A}$};
		 \node (c) at (1,0) {$\complx C$};
		 \node (d) at (2,0) {$\complx A[1]$};
		 \draw[->] (a) to node[above]{$ $} (b);
		 \draw[->] (b) to node[above]{$ $} (c);
		 \draw[->] (c) to node[above]{$ $} (d);
		\end{tikzpicture}
      \end{center}
      Per ogni $n$ quindi si ha la successione esatta
      \[
	0\rightarrow A^n \rightarrow I^n \rightarrow C^n \rightarrow 0
      \]
      allora per il lemma precedente i $C^n$ sono oggetti adatti. Dalla sequenza esatta lunga in 
      comologia otteniamo anche che il complesso $\complx{C}$ è aciclico. \\
      Applicando $F$:
      \[
	0\rightarrow FA^n \rightarrow FI_A^n \rightarrow FC^n \rightarrow 0
      \]
      è esatta.\\
      Vogliamo mostrare che $F\complx{A}\rightarrow F\complx{I}_A$ è un quasi isomorfismo. Ma dato che $\complx C$ è 
      fatto di oggetti adatti e è aciclico per il Lemma \ref{lemma:adatti_aciclici}.3 $F\complx C$ è aciclica e quindi la successione esatta lunga in comologia è
      \[
	\dots \rightarrow H^i(FA) \rightarrow H^i(FI) \rightarrow 0 \rightarrow H^{i+1}(FA) \rightarrow H^{i+1}(FI)\rightarrow 0 \rightarrow \dots
      \]
      Allora $F(\complx{A})= F(\complx{I}_A)=RF(\complx{A})$ in $\der{\B}^+$, che è quello che volevamo.
      \end{proof}

%       \begin{oss}{Da sistemare}
%  Se $F\colon\A\rightarrow\B$ e $ G\colon\B\rightarrow\mathcal{C}$ sono due funtori esatti a sinistra, abbiamo visto che sono bene definiti $RF$ e $ RG$.
%  Risulta lecito chiedersi che cosa accade componendoli, in generale non \`e vero che $RG\circ RF = R(GF)$. Approfondiremo qui questo fatto.\\
%  
%  \begin{prop}
%   Supponiamo $RG\circ RF = R(GF)$. (?)
%   Se $I\in \Ob \A$ \'e iniettivo allora $F(I)$ \`{e} adatto per $G$.
%  \end{prop}
%  \begin{proof}\marginpar{???}
%   $I$ \`e adatto e dunque $RF(I)=F(I)$. $RG(F(I))=R(GF)(I)$
%  \end{proof}
%        
%       \end{oss}
      
 \subsection{Il funtore Tor}

  
  Studiamo adesso il funtore derivato $\Tor_A(M,\_)$, con $M$ un $A$ modulo, rispetto $F=M\otimes_A \_ $.
  \begin{prop}
  Se $N$ \`e un $A$ modulo proiettivo allora $\Tor^i(M,N)=0$ per ogni $i>0$.
  \end{prop}
  \begin{proof}
   $N$ \`e proiettivo e quindi $\complx{\underline N}$ \`e una sua risoluzione proiettiva. Applicando
   $F$ otteniamo il complesso
   \[
	0\rightarrow 0 \rightarrow N\otimes_A M  \rightarrow 0 \rightarrow 0
      \]
   la cui comologia \`e zero per $i\neq 0$.
  \end{proof}

  Vale anche la proprietà simmetrica:
  \begin{prop}
  Se $M$ \`e un $A$ modulo piatto allora $\Tor^i(M,N)=0$ per ogni $i>0$.
  \end{prop}
 \begin{proof}
  Sia $\complx{P}\in\Com^+$ \`e una risoluzione proiettiva di $N$.
  \[
	\dots \rightarrow P^{-2}\rightarrow P^{-1} \rightarrow P^{0} \rightarrow 0
   \]
   questo complesso \`e esatto ovunque tranne che in zero e quindi tensorizzando per $M$, che \`e piatto,
   l'esattezza si conserva per $i<0$, che equivale a dire che$\Tor^i(M,N)=0$ per ogni $i>0$.
 \end{proof}
  
  
 Sappiamo che $\Tor_A^0(M,N)=M\otimes_A N=N\otimes_A M = \Tor_A^0(N,M)$, in effetti si ha che questa simmetria \`e sempre valida.
 Vediamo alcuni fatti e osservazioni che ci aiuteranno a dimostrarlo:
 
 \begin{oss}
    Consideriamo una sequenza esatta corta
    \begin{center}
    \begin{tikzpicture}[scale=1]
	 \node (a) at (-1,0) {$0$};
 	 \node (b) at (0,0) {$A$};
	 \node (c) at (1,0) {$B$};
 	 \node (d) at (2,0) {$C$};
 	 \node (a3) at (3,0) {$0$};
	 \draw[->] (a) -- (b);
	 \draw[->] (b) to node[above]{$ $} (c);
	 \draw[->] (c) to node[above]{$ $} (d);
	 \draw[->] (d) -- (a3);
       \end{tikzpicture}
      \end{center} 
      applicando $F$ troviamo la sequenza esatta lunga
      \begin{center}
	\begin{tikzpicture}[scale=2]
	 \node (a) at (-1,0) {$\dots$};
	 \node (b) at (0,0) {$\Tor^1(M,C)$};
	 \node (c) at (1,0) {$M\otimes A$};
	 \node (d) at (2,0) {$M\otimes B$};
	 \node (e) at (3,0) {$M\otimes C$};
	 \node (f) at (3.7,0) {$0$};
	 \draw[->] (a) -- (b);
	 \draw[->] (b) -- (c);
	 \draw[->] (c) -- (d);
	 \draw[->] (d) -- (e);
	 \draw[->] (e) -- (f);
   \end{tikzpicture}
  \end{center} 
  Se invece abbiamo 
    \begin{center}
    \begin{tikzpicture}[scale=1]
	 \node (a) at (-1,0) {$0$};
 	 \node (b) at (0,0) {$M_1$};
	 \node (c) at (1,0) {$M_2$};
 	 \node (d) at (2,0) {$M_3$};
 	 \node (a3) at (3,0) {$0$};
	 \draw[->] (a) -- (b);
	 \draw[->] (b) to node[above]{$f$} (c);
	 \draw[->] (c) to node[above]{$g$} (d);
	 \draw[->] (d) -- (a3);
       \end{tikzpicture}
      \end{center} 
      e definiamo $F_i(X)\coloneqq M_i\otimes_A X$. Allora sono definiti dei morfismi
      \[F_1(X)\xrightarrow{f\otimes id}F_2(X)\]
      \[F_2(X)\xrightarrow{g\otimes id}F_3(X)\]
      e se $M$ \`e proiettivo (piatto) allora
      \begin{center}
	\begin{tikzpicture}[scale=2]
	 \node (b) at (0,0) {$0$};
	 \node (c) at (1,0) {$M_1\otimes X$};
	 \node (d) at (2,0) {$M_2\otimes X$};
	 \node (e) at (3,0) {$M_3\otimes X$};
	 \node (f) at (4,0) {$0$};
	 \draw[->] (b) -- (c);
	 \draw[->] (c) -- (d);
	 \draw[->] (d) -- (e);
	 \draw[->] (e) -- (f);
   \end{tikzpicture}
  \end{center} 
  \`e esatta.
 \end{oss}

 \begin{lemma}
  Siano $F,G,H\colon \A \rightarrow \B$ funtori esatti a destra e una successione di funtori
  \begin{center}
    \begin{tikzpicture}[scale=1]
	 \node (a) at (-1,0) {$0$};
 	 \node (b) at (0,0) {$F$};
	 \node (c) at (1,0) {$G$};
 	 \node (d) at (2,0) {$H$};
 	 \node (a3) at (3,0) {$0$};
	 \draw[->] (a) -- (b);
	 \draw[->] (b) to node[above]{$ $} (c);
	 \draw[->] (c) to node[above]{$ $} (d);
	 \draw[->] (d) -- (a3);
       \end{tikzpicture}
      \end{center} 
      esatta sui proiettivi con $\A $ \`e una categoria con abbastanza proiettivi e iniettivi. Allora
      
      \[
	\dots \rightarrow L^1G(X)\rightarrow  L^1H(X) \rightarrow F(X) \rightarrow G(X)\rightarrow H(X) \rightarrow 0
      \]
    \`e esatta.
 \end{lemma}
 \begin{proof}
  Sia $\complx{P}_X$ una risoluzione proiettiva di $\complx{\underline X}$, in ogni grado $n$
    \begin{center}
    \begin{tikzpicture}[scale=1.8]
	 \node (a) at (-1,0) {$0$};
 	 \node (b) at (0,0) {$F(P^n_X)$};
	 \node (c) at (1,0) {$G(P^n_X)$};
 	 \node (d) at (2,0) {$H(P^n_X)$};
 	 \node (a3) at (3,0) {$0$};
	 \draw[->] (a) -- (b);
	 \draw[->] (b) to node[above]{$ $} (c);
	 \draw[->] (c) to node[above]{$ $} (d);
	 \draw[->] (d) -- (a3);
       \end{tikzpicture}
      \end{center} 
      \`e esatta per ipotesi. \\
      Il tringolo 
      	 \begin{center}
	       \begin{tikzpicture}[scale=1.5]
		 \node (a) at (-1,0) {$F\complx{P}$};
		 \node (b) at (0,0) {$G\complx{P}$};
		 \node (c) at (1,0) {$H\complx{P}$};
		 \node (d) at (2,0) {$F\complx{P}[1]$};
		 \draw[->] (a) to node[above]{$ $} (b);
		 \draw[->] (b) to node[above]{$ $} (c);
		 \draw[->] (c) to node[above]{$ $} (d);
		\end{tikzpicture}
	    \end{center}
	    allora \`e distinto.\\
       Applicando la successione esatta lunga si ha la tesi.\\
  \end{proof}

  \begin{cor}
  Presa la seguente successione esatta di moduli
  \begin{center}
    \begin{tikzpicture}[scale=1]
	 \node (a) at (-1,0) {$0$};
 	 \node (b) at (0,0) {$M_1$};
	 \node (c) at (1,0) {$M_2$};
 	 \node (d) at (2,0) {$M_3$};
 	 \node (a3) at (3,0) {$0$};
	 \draw[->] (a) -- (b);
	 \draw[->] (b) to node[above]{$f$} (c);
	 \draw[->] (c) to node[above]{$g$} (d);
	 \draw[->] (d) -- (a3);
       \end{tikzpicture}
      \end{center} 
      Allora, preso un modulo $X$ qualsiasi, la sequenza lunga
      \[
	\dots \rightarrow \Tor^1(M_2,X)\rightarrow  \Tor^1(M_1,X) \rightarrow  M_1\otimes X  \rightarrow  M_2\otimes X \rightarrow M_3\otimes X \rightarrow 0
      \]
      \`e esatta.
  \end{cor}\label{cor:esattainlunga}
  \begin{proof}
  I funtori $F_i(X)\coloneqq M_i\otimes_A X$ per $i=1,2,3$ dell'osservazione verificano le ipotesi del lemma.
  \end{proof}
  
  Quanto appena detto \`e sufficiente per mostrare la simmetria voluta, ossia: 
  
 \begin{prop}
  Siano $M,N$ due $A$ moduli qualsiasi. Allora $\Tor_A^i(M,N)=\Tor_A^i(N,M)$ per ogni $i$.
 \end{prop}
 \begin{proof}
  Per calcolare $\Tor_A^i(M,N)$, prendiamo una risoluzione proiettiva $\complx{P}$ di $N$
  e gli applichiamo $M\otimes \_$. Chiamiamo adesso $N'$ il nucleo della proiezione di $P^0$ su $N$; la seguente successione allora \`e esatta:
  \begin{center}
    \begin{tikzpicture}[scale=1]
	 \node (a) at (-1,0) {$0$};
 	 \node (b) at (0,0) {$N'$};
	 \node (c) at (1,0) {$P^0$};
 	 \node (d) at (2,0) {$N$};
 	 \node (a3) at (3,0) {$0$};
	 \draw[->] (a) -- (b);
	 \draw[->] (b) to node[above]{$ $} (c);
	 \draw[->] (c) to node[above]{$ $} (d);
	 \draw[->] (d) -- (a3);
       \end{tikzpicture}
      \end{center}
      Abbiamo per il Corollario \ref{cor:esattainlunga} con $M\otimes \_$ la successione esatta lunga
      \begin{equation}\label{eq:tor1}
	\begin{matrix}
	\dots\Tor^1(M,N)\rightarrow  M\otimes N' \rightarrow M \otimes P^0 \rightarrow M \otimes N \rightarrow 0\\
	\dots\rightarrow 0 \rightarrow \Tor^2 (M,N)\rightarrow \Tor^1(M,N') \rightarrow 0 \rightarrow  \dots
	\end{matrix}
      \end{equation}
      Dove gli zeri compaiono al posto di $\Tor^i(M,P^0)$ dato che $P^0$ \`e proiettivo. 
      Allora $\Tor^i(M,N) = \Tor^{i-1}(M,N')$ per ogni $i\geq 2$. \\
      Analogamente col funtore $\_\otimes M$ otteniamo:
      \begin{equation}\label{eq:tor2}
	\begin{matrix}
	\dots\Tor^1(N,M)\rightarrow  N'\otimes M  \rightarrow P^0 \otimes M \rightarrow N \otimes M\rightarrow 0\\
	\dots\rightarrow 0 \rightarrow \Tor^2 (N,M)\rightarrow \Tor^1(N',M) \rightarrow 0 \rightarrow  \dots
	\end{matrix}
      \end{equation}
      e quindi $\Tor^i(N,M) = \Tor^{i-1}(N',M)$ per ogni $i\geq 2$.\\
       Osserviamo che essendo la risoluzione aciclica in $P^{-1}$ abbiamo che il complesso
       \[
 	\dots\rightarrow P^{-3} \rightarrow P^{-2} \rightarrow P^{-1} \rightarrow 0\rightarrow \dots
       \]
       \`e una risoluzione per $N'$. \\
       Mostriamo per induzione su $i$ la tesi.
      \begin{itemize}
       \item [$i=0$] È noto che $M \otimes N= N\otimes M$.
       \item [$i=1$] Dal caso $i=0$ e dalla prime parte delle successioni 
		      \ref{eq:tor1} e \ref{eq:tor2} si ha che allora $\Tor_A^1(M,N)=\Tor_A^1(N,M)$.
       \item [$i\geq 1$]Per ipotesi induttiva $\Tor_A^{i-1}(M,N')=\Tor_A^{i-1}(N',M)$ e dunque per quanto
	      provato prima $\Tor_A^i(M,N)=\Tor_A^{i-1}(M,N')=\Tor_A^{i-1}(N',M)=\Tor_A^i(N,M)$.
      \end{itemize}
 \end{proof}
 \begin{prop}\label{prop:piattotor}
  I seguenti fatti sono equivalenti:
  \begin{enumerate}[(1)]
   \item $M$ è un $A$ modulo piatto;
   \item $\Tor_A^i(M,N)=0$ per ogni $N$ e per ogni $i>0$;
   \item $\Tor_A^1(M,N)=0$ per ogni $N$.
  \end{enumerate}
 \end{prop}
 \begin{proof}
  Ovviamente (1) $\Rightarrow$ (2) $\Rightarrow$ (3) .\\
  Supponiamo che valga (3) e consideriamo una successione esatta
      \begin{center}
      \begin{tikzpicture}[scale=1]
	 \node (a) at (-1,0) {$0$};
 	 \node (b) at (0,0) {$A$};
	 \node (c) at (1,0) {$B$};
 	 \node (d) at (2,0) {$C$};
 	 \node (a3) at (3,0) {$0$};
	 \draw[->] (a) -- (b);
	 \draw[->] (b) to node[above]{$ $} (c);
	 \draw[->] (c) to node[above]{$ $} (d);
	 \draw[->] (d) -- (a3);
       \end{tikzpicture}
      \end{center} 
   Allora anche la successione 
   \begin{center}
      \begin{tikzpicture}[scale=1.8]
	 \node (a) at (-0.8,0) {$0$};
 	 \node (b) at (0,0) {$M\otimes A$};
	 \node (c) at (1,0) {$M\otimes B$};
 	 \node (d) at (2,0) {$M\otimes C$};
 	 \node (a3) at (2.8,0) {$0$};
	 \draw[->] (a) -- (b);
	 \draw[->] (b) to node[above]{$ $} (c);
	 \draw[->] (c) to node[above]{$ $} (d);
	 \draw[->] (d) -- (a3);
       \end{tikzpicture}
      \end{center} 
    è esatta e quindi $M$ è piatto.
 \end{proof}

 \chapter{Dimensione comologica di anelli noetheriani locali}
  I risultati che abbiamo ottenuto riguardo $\Tor$ trovano applicazione nello studio degli anelli noetheriani locali
  e, per  analogia, negli anelli graduati. A meno di specificare, in questo paragrafo indicheremo con $(A,\m)$ un anello noetheriano locale
  con massimale $\m$ e $M$ un $A$ modulo finitamente generato, inoltre con $k$ indicheremo il campo residuo $A/\m$.  
  
  \begin{oss}
   I risulti che presenteremo continuano ad essere validi sostituendo nelle ipotesi $A$ anello graduato con $A_0$ campo e $M$ modulo graduato finitamente generato.
  \end{oss}
  
  In primo luogo ci interessa capire come sono fatte le risoluzioni libere (proiettive/piatte) di $M$. Dato che \`e finitamente generato, possiamo definire
  $n_0$ il minimo numero di generatori di $M$ e sappiamo che esiste un omomorfismo di moduli surgettivo tra $F^0\coloneqq A^{n_0}$ e $M$:
  \[
   F^0\xrightarrow{\partial^{0}} M\rightarrow 0
  \]
  A partire da qui, iterativamente, possiamo costruire una risoluzione libera di $M$: indichiamo con $n_i$ il minimo numero di generatori di $\ker \partial^{-i}$
  e con $F^{-i}\coloneqq A^{n_i}$.
  \begin{de}
  Una risoluzione
  \[
    \dots\xrightarrow{\partial^{-4}}F^{-3}\xrightarrow{\partial^{-3}} F^{-2}\xrightarrow{\partial^{-2}} F^{-1}\xrightarrow{\partial^{-1}}F^0\xrightarrow{\partial^{0}} M\rightarrow 0
  \]
  costruita come sopra \`e detta \emph{risoluzione libera minimale}.
  \end{de}

  \begin{lemma}\label{bordinulli}
   Una risoluzione libera di $M$ \`e minimale se e solo se il complesso tensorizzato per $k$ ha tutti i bordi nulli eccetto in zero, ossia  
   $\ \bar\partial^{-i}\coloneqq \partial^{-i}\otimes id_k=0 $ per $i> 0$.
  \end{lemma}
  \begin{proof}
   Supponiamo che
   \[
    \dots\xrightarrow{\partial^{-3}} F^{-2}\xrightarrow{\partial^{-2}} F^{-1}\xrightarrow{\partial^{-1}}F^0\xrightarrow{\partial^{0}} M\rightarrow 0
  \]
  sia una risoluzione libera minimale. Applichiamo il funtore $\_\otimes k$:
  \[
    \dots\xrightarrow{\bar\partial^{-3}} F^{-2}\otimes k\xrightarrow{\bar\partial^{-2}} F^{-1}\otimes k
    \xrightarrow{\bar\partial^{-1}}F^0\otimes k\xrightarrow{\bar\partial^{0}} M\otimes k\rightarrow 0
  \]
  Osserviamo\footnote{Nakayama} che  $M\otimes k\simeq k^{n_0}$e $F^0\otimes k\simeq k^{n_0}$ e quindi $\bar\partial^{0}$ \`e un isomorfismo (sono spazi vettoriali della stessa dimensione
  e il bordo \`e suriettivo), cosicché $\bar\partial^{-1}=0$.\\
  Supponiamo adesso $\bar\partial^{-j}$ sia zero per $0<j<i$. Chiamiamo $M'$ il conucleo della mappa $\partial^{-i}$: 
  \[
   F^{-i}\xrightarrow{\partial^{-i}}F^{-i+1}\xrightarrow{\beta} M'\rightarrow 0
  \]
  Se tensorizziamo questa sequenza esatta abbiamo
  \[
   F^{-i}\otimes k\xrightarrow{\bar\partial^{-i}}F^{-i+1}\otimes k\xrightarrow{\bar\beta} M'\otimes k\rightarrow 0
  \]
  Dato che il complesso \`e esatto $M'$ \`e anche il nucleo di $\partial^{-i+1}$ e quindi dato che \`e minimale $M'\otimes k=k^{n_i}$.
  Ci siamo ricondotti al caso base e quindi la tesi e vera per ipotesi induttiva.\\
  Viceversa, supponiamo che $\bar\partial^{-i}=0 $ per $i> 0$. Allora $\bar\partial^0$ \`e iniettiva e dunque
  \[
   n_0\leq\dim_k  (M\otimes k)=\min\set{\#\text{ generatori di }M}\leq n_0
  \]
  e quindi $n_0=\min\set{\#\text{ generatori di }M}$. Analogamente a come abbiamo fatto prima, sfruttando l'esattezza, ci si pu\`o 
  ricondurre sempre a questo caso. In tal modo otteniamo che  $n_i=\min\set{\#\text{ generatori di }\ker\partial^{-i}}$ per ogni $i$, che equvale a dire che la risoluzione libera 
  presa \`e minimale.
  \end{proof}
  
  Questo risulto ci permette di dimostrare un risultato molto importante a proposito della dimensione comologica proiettiva dei moduli finitamente generati su anelli 
  locali noetheriani.
  
  \begin{teo}\label{teo:dhploc}
  Sia $(A,\m)$ un anello noetheriano locale con $k$ campo residuo di $\m$ e $M$ un $A$ modulo finitamente generato. Siano poi 
  $n=\dhp M$ e $d$ la lunghezza di una risoluzione libera minimale di $M$. Allora $n=d$ e 
  \[
   \Tor^i_A(M,k)
   \begin{cases}
    =0 & i>d;\\
    \neq 0 & i\leq d,\\
   \end{cases}
  \]
  Inoltre se 
  \[
    \dots\xrightarrow{\partial^{-3}} F^{-2}\xrightarrow{\partial^{-2}} F^{-1}\xrightarrow{\partial^{-1}}F^0\xrightarrow{\partial^{0}} M\rightarrow 0
  \]
  \`e una risoluzione libera minimale $n_i=\dim_k \Tor^i_A(M,k)$.
  \end{teo}

  \begin{oss}
   $\Tor^i_A(M,k)$ \`e un $k$-spazio vettoriale. Infatti $\Tor^i_A(M,k)=\Tor^i_A(k,M)$ che \`e la comologia di una risoluzione proiettiva o 
   piatta di $M$ tensorizzata per $k$, i cui bordi sono proprio mappe di spazi vettoriali.
  \end{oss}

  \begin{proof}\*
   $d\geq n$ infatti ogni modulo libero \`e proiettivo. Se considerariamo poi  risoluzione proiettiva di $M$, per calcolare i $\Tor$ tensorizziamo per $k$
       \[ 
 	0 \rightarrow P^{-n}\rightarrow \dots \rightarrow P^{-2} \rightarrow P^{-1} \rightarrow  P^{0} \rightarrow 0 \rightarrow \dots
       \]
       allora chiaramente  $\Tor^i_A(k,M)=0$ per $i>n$.\\
       Se 
       \[
      \dots\xrightarrow{\partial^{-3}} F^{-2}\xrightarrow{\partial^{-2}} F^{-1}\xrightarrow{\partial^{-1}}F^0\xrightarrow{\partial^{0}} M\rightarrow 0
	\]
       \`e una risoluzione libera minimale di $M$ lunga $s$ applicando $\_\otimes k$ per il Lemma \ref{bordinulli} otteniamo un complesso con i bordi tutti nulli: passando
       alla comologia quindi abbiamo che $\Tor^i_A(k,M)=F^{-i}\otimes k=k^{n_i}$ e $n_i=\dim_k\Tor^i_A(k,M)$. Inoltre
       \[
	  \Tor^i_A(M,k)
	  \begin{cases}
	    =0 & i>s;\\
	    \neq 0 & i\leq s,\\
	  \end{cases}
	\]
	In genera vale che $s\geq d\geq n$ ma 
	\[
	  \Tor^i_A(M,k)
	  \begin{cases}
	    =0 & i>n;\\
	    \neq 0 & i\leq s,\\
	  \end{cases}
	\]
	e quindi $n+1\geq s+1$, ossia $n\geq s\geq d\geq n$. 
  \end{proof}
  
  \begin{oss}
   Il teorema ci dice che le risoluzioni libere minimali sono anche di lunghezza minima.
  \end{oss}

  \begin{cor}
   Sia $(A,\m)$ un anello noetheriano locale con $k$ campo residuo di $\m$ e $M$ un $A$ modulo finitamente generato. I seguenti fatti sono equivalenti:
   \begin{enumerate}[(1)]
    \item $M$ libero
    \item $M$ proiettivo
    \item $M$ piatto
   \end{enumerate}
  \end{cor}
  \begin{proof}
  
   (1)$\iff$(2) \`e Lemma \ref{loc:proiettivoimplicalibero}.\\
   (2)$\iff$(3) noto.\\
   (3)$\iff$(2) Per la Proposizione \ref{prop:piattotor}, $\Tor^1_A(k,M)=0$ perciò una risoluzione libera minimale \`e lunga zero e quindi $M$ \'e libero.
  \end{proof}
   
  \begin{cor}
   Sia $A$ un anello noetheriano e $M$ un $A$ modulo finitamente generato.
   I seguenti fatti sono equivalenti:
   \begin{enumerate}[(1)]
    \item $M$ localmente libero.
    \item $M$ proiettivo
    \item $M$ localmente piatto
   \end{enumerate}
  \end{cor}
  \begin{proof}(1)$\iff$(2) \`e  il Teorema \ref{teo:loclibero}.\\
  $M$ proiettivo $\iff$ $M$ localmente piatto $\iff$ $M$ localmente libero.
  \end{proof}

   Se $(A,\m)$ è un anello locale noetheriano, il Teorema \ref{teo:dhploc} mette in relazione la lunghezza di 
   una risoluzione libera minimale di un modulo
   finitamente generato $M$ con i $\Tor^i (k,M)$, dove $k$ è il campo residuo di $\m$. Abbiamo mostrato tuttavia che
   il funtore derivato $\Tor$ è simmetrico nelle entrate, diventa quindi interessante lo studio delle risoluzioni dell' $A$ modulo $k$.
   In questo paragrafo vedremo la costruzione, a tale scopo, del complesso di Koszul. Ci serve tuttavia introdurre alcune nozioni e risultati
   preliminari.\\
   
   \subsection{Prodotto esterno}
    
    \begin{de}
    Sia $A$ un anello commutativo con identità e $M$ un $A$ modulo. Diremo che 
    \[\Phi\colon M^k \longrightarrow \wedge^k M  \]
    è il \emph{prodotto esterno} se 
    \begin{enumerate}
     \item $\Phi$ è multilineare
     \item $\Phi$ è alternante, ossia $\Phi(m_1,\dots, m,m, \dots m_k)=0$ (antisimmetrico).
     \item Per ogni $\Psi \colon M^k \rightarrow U $ multilineare alternante esiste unico un omomorfismo di moduli
	    $\Omega\colon \wedge^k M \rightarrow U$ tale che $\Psi=\Omega \circ \Phi$.
    \end{enumerate}
    \end{de}
    \begin{oss}
     Sia $\text{char} k \neq 2$. Se $\varphi$ è un applicazione alternante allora 
     $\varphi(m_1, \dots, m_k)=\varepsilon(\sigma)\varphi(m_{\sigma 1}, \dots, m_{\sigma k})$ con $\sigma \in S_k$.
    \end{oss}

    \begin{prop}
     Il prodotto esterno esiste.
    \end{prop}
    \begin{proof}
     Esiste una mappa 
     \[
     \begin{matrix}
      \pi \colon& M^k&\longrightarrow & M^{\otimes^ k}\\
		& (m_1, \dots, m_k) & \longmapsto &  m_1\otimes \dots\otimes m_k  
     \end{matrix}    
     \]
      Definiamo allora $\wedge^kM$ come $M^{\otimes^ k}$ modulo il gruppo generato dai podotti in cui compaiono due entrate
      uguali e $p$ la proiezione. Allora $\Phi=p\circ\pi$ . Le prime due proprietà del prodotto esterno valgono per costruzione;
      verifichiamo la terza: prendiamo $\Psi \colon M^k \rightarrow U $ alternante e multilineare,
      per la proprietà universale del prodotto tensore esiste $\Omega_1$  tale che $\Psi=\Omega_1\circ \pi$ e poichè è alternante
      passa a quoziente.
    \end{proof}

    \begin{oss}\
    \begin{itemize}
     \item Indicheremo $\Phi (m_1, \dots, m_k)= m_1\wedge \dots\wedge m_k.$
     \item Se $M=A^n$, detta $\set{e_1,\dots, e_n}$ la base canonica,  per $I=\set{i_1<\dots i_k\leq n}$ definiamo
	    $e_I\coloneqq e_{i_1}\wedge\dots\wedge e_{i_k}$. Allora\footnote{Non lo dimostriamo.} 
	    \[
	     \set{e_I  \ \colon \ \#I=k}
	    \]
	  è una base di $\wedge^kM$.
     \item Sia $M=A^n$. Allora 
	    \[
	     \begin{cases}
	      \wedge^0 M\coloneqq A\\
	      \wedge^n M=A\\
	      \wedge^k M=0 & \text{ per } k>n \\
	     \end{cases}
	    \]
     \item Sia $T\colon M\rightarrow N$ un omomorfismo di moduli, allora
		tramite $T^k\colon M^k\rightarrow N^k$ è indotto un omomorfismo $\wedge^k T\colon \wedge^k M\rightarrow \wedge^k N$.
		Se $k=n$ è proprio la moltiplicazione per $\det (T)$.
    \end{itemize}
    \end{oss}   

   \section{Complesso di Koszul}
    Il nostro obiettivo è quello di costruire una risoluzione libera di $k=A/\m$, per $A$ anello noetheriano locale.
    Per fare questo prima definiamo le seguenti nozioni che valgono per $A$ anello qualsiasi:
    \begin{de}
     $x_1,\dots,x_n\in A$ si dice \emph{successione regolare} se $x_1$ non è un divisore di zero in $A$ e 
     $x_i$ non è un divisore di zero in $\faktor {A}{(x_1,\dots,x_{i-1})}$ per ogni $i$.
    \end{de}
    \noindent
    Prendiamo una successione regolare di $x_1,\dots,x_m\in A$  e indichiamo con $M=A^m$, allora
    $\underline{x}=(x_1,\dots,x_m)\in M$. Il \emph{complesso di Koszul} $\complx K (\underline x)$ è
    
    \[
    \dots 0 \xrightarrow{\partial_K^{-1}} \wedge^0 M \xrightarrow{\partial_K^{0}} \wedge^1 M \xrightarrow{\partial_K^{1}}\wedge^2 M
    \xrightarrow{\partial_K^{2}}\dots\xrightarrow{\partial_K^{m-1}}\wedge^m M\rightarrow 0\dots
    \]
    con $\partial_K^i=\_\wedge \underline{x}$.\\
    Affinché questo sia davvero un complesso c'è da verificare che $\partial_K^i\circ \partial_K^{i-1}=0$, ma è ovviamente vero
    poiché il prodotto esterno è alternante e quindi
    $\partial_K^i\circ \partial_K^{i-1}(y)=y\wedge \underline{x}\wedge \underline{x}=0$.\\
    
    Per questo complesso vale una condizione necessaria e sufficiente sulla comologia legata alla scelta di $\underline{x}$,
    di cui noi mostreremo però solo la necessità.
    \begin{teo}\label{teo:koszul}
     Se $x_1,\dots,x_m\in A$ è una successione regolare, allora 
     \[
      H^i(\complx K (\underline x))=
      \begin{cases}
       0 & \text{ se } i\neq m \\
       \faktor{A}{(x_1,\dots, x_m)}& \text{ se } i= m \\
      \end{cases}
     \]
    Viceversa se $H^i(\complx K (\underline x))\neq0$ solo in grado $m$ la successione è regolare.
    \end{teo}
    
    \begin{proof}
    Dimostriamo la tesi per induzione su $m$.
    \begin{itemize}
     \item [$m=1$] Dato che $M=A$ allora il complesso è
	    \[
	    \dots 0 \xrightarrow{\partial_K^{-1}} A \xrightarrow{\partial_K^{0}} A \xrightarrow{\partial_K^{1}} 0
	    \rightarrow 0\dots
	    \]
	    con $\partial_k=\cdot x_1$.\\ Dato che $x_1\nmid 0$,  $\partial_K^0$ è iniettiva e quindi
	    $H^0=0$ e $H^1=\faktor A {(x_1)}$.
     \item [$m\Rightarrow m+1$] Indichiamo con $\complx{K}_m$ il complesso di Koszul ottenuto da $M_m=A^m$ e
	     $x_1,\dots,x_m\in A$, una successione regolare, e con $\complx{K}_{m+1}$ il complesso di Koszul
	     ottenuto da $M_{m+1}=M_m\oplus A\varepsilon=A^{m+1}$ con la successione regolare $x_1,\dots,x_{m+1}$.\\
	     Allora $\wedge^0M_{m+1}=\wedge^0M_{m+1}=A$ e $\wedge^1M_{m+1}=M_{m+1}=M_m\oplus A\varepsilon$,
	     per $1<k<m+1$ abbiamo che 
	     $\wedge^kM_{m+1}=\wedge^kM_{m}\oplus (\wedge^{k-1}M_{m}\wedge  A\varepsilon)=A^{m+1 \choose k}$.\\
	     Infatti abbiamo che una base\footnote{Questi sono algebricamente indipendenti
	     per costruzione, inoltre sono in numero esatto poichè ${m+1 \choose k}={m\choose k}+{m\choose k-1}$.}
	     di  $\wedge^kM_{m+1}$ è data dagli $e_I$, con $I=\set{i_1<\dots i_k\leq m}$, più gli $e_J\wedge\varepsilon$, con $J=\set{i_1<\dots i_{k-1}\leq m}$. \\
	     Dobbiamo descrivere adesso il bordo $\partial$ del complesso. Ricordando come abbiamo definito il complesso di Koszul,
	     indicando con $\underline x(m)=(x_1,\dots,x_m)$, per $e_I\in \wedge^kM_{m+1}$ poniamo sui generatori della prima forma
	     \[
	      \partial_{m+1}^k(e_{i_1}\wedge\dots\wedge e_{i_k})= e_{i_1}\wedge\dots\wedge e_{i_k}\wedge(\underline x(m)+ x_{m+1}\varepsilon)
	     \]
	     e usando la multilinearità
	     \begin{equation*}
	      \partial_{m+1}^k(e_{i_1}\wedge\dots\wedge e_{i_k})=\partial^k_m(e_{i_1}\wedge\dots\wedge e_{i_k})+
	      x_{m+1}(e_{i_1}\wedge\dots\wedge e_{i_k}\wedge \varepsilon)
	     \end{equation*}
	     Preso un generatore della seconda forma invece
	      \begin{align*}
	      \partial_{m+1}^k(e_{i_1}\wedge\dots\wedge e_{i_{k-1}}\wedge\varepsilon)
		    &= e_{i_1}\wedge\dots\wedge e_{i_{k-1}}\wedge\varepsilon\wedge(\underline x(m)+ x_{m+1}\varepsilon)\\
		    &= e_{i_1}\wedge\dots\wedge e_{i_{k-1}}\wedge\varepsilon \wedge\underline x(m)\\
		    &= - e_{i_1}\wedge\dots\wedge e_{i_{k-1}} \wedge\underline x(m)\wedge\varepsilon\\
		    &= -\partial_{m}^k(e_{i_1}\wedge\dots\wedge e_{i_{k-1}}).
	      \end{align*}
	     In altri termini
	     \[
	     \partial_{m+1}^k=
	      \begin{pmatrix}
	       \partial^k_m & 0\\
	       x_{m+1} & -\partial_{m}^k
	      \end{pmatrix}
	     \]
	      Abbiamo appena mostrato che, a meno del segno e di shiftare per uno, $\complx K_{m+1}=\Cono(f)$ per 
	      \[\begin{matrix}
	       f\colon & \complx K_{m}	&\longrightarrow& \complx K_{m}\\
			& a 		&\longmapsto	&  -x_{m+1}a
	      \end{matrix}\]
	      e dunque $H^i(\complx K_{m+1})\simeq H^{i+1}(\Cono(f))$.\\
	      Abbiamo mostrato che la successione esatta corta
		\begin{center}
		\begin{tikzpicture}[scale=1.5]
		\node (a) at (-0.8,0) {$0$};
		\node (b) at (0,0) {$\complx K_{m}$};
		\node (c) at (1,0) {$\Cono(f)$};
		\node (d) at (2.2,0) {$\complx K_{m}[1]$};
		\node (a3) at (3,0) {$0$};
		\draw[->] (a) -- (b);
		\draw[->] (b) to node[above]{$ $} (c);
		\draw[->] (c) to node[above]{$ $} (d);
		\draw[->] (d) -- (a3);
		\end{tikzpicture}
		\end{center}
	      induce una successione esatta lunga in comologia, usando che per ipotesi induttiva $H^i(\complx K_{m})=0$ per $i\neq m$
	      otteniamo che $H^i(\Cono(f))=0$ per $i<m-1$ e
	      \[
	       0\rightarrow H^{m-1}(\Cono(f))\rightarrow \faktor{A}{I}\xrightarrow{\omega} \faktor{A}{I}
	       \rightarrow H^{m}(\Cono(f))\rightarrow 0
	      \]
	      con $I=(x_1,\dots, x_m)$ e $\omega=-H^i(f)$. Dato che $x_1,\dots, x_m,x_{m+1}$ è regolare, 
	      $\omega$ è la moltiplicazione per un elementi non nullo, allora è iniettiva, quindi
	      $H^{m-1}(\Cono(f))=0$, e $H^{m}(\Cono(f)) =\faktor{A}{(I,x_{m+1})}$. Ricomponendo quanto appena detto allora
	      \[
	      H^i(\complx K_{m+1} (\underline x))=H^{i+1}(\Cono(f))=
	      \begin{cases}
	      0 & \text{ se } i\neq m+1 \\
	      \faktor{A}{(x_1,\dots, x_{m+1})}& \text{ se } i= m+1 \\
	      \end{cases}
	      \]
    \end{itemize}
    \end{proof} 

    \section{Anelli noetheriani regolari}
    Torniamo al caso a cui siamo interessati, $(A,\m)$ un anello locale noetheriano, aggiungendo l'ipotesi che $A$ sia regolare.
    Allora sappiamo che esistono dei generatori del massimale $\m=(x_1,\dots, x_{m})$ tali che
    $m=\dim A$ e $\bar x_1,\dots, \bar x_{m}$ sono una base del $k$-spazio vettoriale $\faktor{\m}{\m^2}$. Questi elementi
    dell'anello godono anche di un'altra proprietà:
    
    \begin{lemma}
     $x_1,\dots, x_{m}$ sono una successione regolare di $A$.
    \end{lemma}
    \begin{proof}
     Procediamo per induzione sulla dimensione dell'anello.
     $A$ è un anello regolare e dunque è un dominio, perciò $x_i\nmid 0$ per ogni $i$. Se $m=1$ allora è ovvio.\\
     Sia $B=A/(x_1)$, allora $\dim B=\dim A - 1=m-1$.
     Sia $p\colon A\twoheadrightarrow B$ la proiezione a quoziente, il massimale di $B$ $\mathfrak n$ è generato 
     da $p(x_2),\dots ,p(x_m)$ e $\faktor{\mathfrak n}{\mathfrak n^2}$ dalle classi di questi elementi. Dato che sono esattamente
     $m-1$ allora sono una base. Per ipotesi induttiva allora sono una successione regolare di $B$, per il terzo teorema d'omomorfismo
     allora abbiamo la tesi.
    \end{proof}

    Consegue immediatamente da questo fatto e dal Teorema \ref{teo:koszul}:
    
    \begin{cor}
     Sia $(A,\m)$ un anello locale noetheriano regolare di dimensione $m$, tale che $\m=(x_1,\dots, x_{m})$. 
     Allora $\complx{K}(\underline{x})$ è un complesso di moduli liberi tali che
     \[
      H^i(\complx K (\underline x))=
      \begin{cases}
       0 & \text{ se } i\neq m \\
       k & \text{ se } i= m \\
      \end{cases}
     \]
     dove $k=\faktor A {\m}$ è proprio il campo residuo. \\
     In particolare  $\complx{K}(\underline{x})$ è una risoluzione libera di $k$.
    \end{cor}

    Da questo otteniamo anche che :
     \begin{cor}
      Per ogni $A$-modulo $M$, con $A$ nelle ipotesi del corollario precedente, vale che
      $\Tor_A^i(k,M)=0$ se $i<0$ e $i>m$.
     \end{cor}

     \begin{proof}
      Basta calcolare la coomologia del complesso di Koszul ottenuto con i generatori dei massimali che ha proprio lunghezza $m+1$.
     \end{proof}

     
    \begin{cor}\label{cor:regfingen}
     Sia $(A,\m)$ anello noetheriano locale regolare e $\m=(x_1,\dots, x_{m})$. Ogni $A$ modulo finitamente generato
     \begin{itemize}
      \item ha una risoluzione libera lunga al più $m$;
      \item $\dhp M\leq m$.
     \end{itemize}
    \end{cor}

    \begin{proof}
     Dato che $\Tor_A^i(k,M)=0$ se $i<0$ e $i>m$, il Teorema \ref{teo:dhploc} ci dice che la minima lunghezza di una risoluzione libera
     di $M$ finitamente generato, che è anche la dimensione comologica proiettiva, deve essere minore di $m$. Naturalmente quindi 
     prendendo una risoluzione libera minimale questa sarà lunga al più $m$.
    \end{proof}

     Sia $(A,\m)$ un anello noetheriano locale regolare di dimensione $n$. Abbiamo visto che $\complx{K}(\underline{x})$, con 
     $(\underline{x})=(x_1,\dots,x_n)=\m$, è una risoluzione libera di $k$. 
     
     \begin{lemma}
      $\complx{K}(\underline{x})$ è una risoluzione libera minimale di $k$.
     \end{lemma}
     \begin{proof}
     Ricordiamo che  $\complx{K}(\underline{x})$ è
     \[
    \dots 0 \xrightarrow{\partial_K^{-1}} \wedge^0 A^n \xrightarrow{\partial_K^{0}} \wedge^1 A^n \xrightarrow{\partial_K^{1}}\wedge^2 A^n
    \xrightarrow{\partial_K^{2}}\dots\xrightarrow{\partial_K^{n-1}}\wedge^n A^n\rightarrow 0\dots
    \]
      con $\partial=\_\wedge \underline{x}$.\\
      Per dire che è minimale usiamo il Lemma \ref{bordinulli}: se applichiamo $\_\otimes k$
      il bordo diventa $\partial^s\otimes id_k$, ma per ogni $v\in\wedge^s A^n$  $\partial^s(v)= v\wedge \underline{x}\in \m A^n$, ricordando che $k=A/\m$, 
      abbiamo allora che $\partial(v)\otimes k= 0$.
     \end{proof}

     \begin{cor}
      La dimensione comologica proiettiva di $k=A/\m$ è proprio $n$.
     \end{cor}    
     
     In realtà vale qualcosa di molto più forte:
     \begin{teo}
      Sia $(A,\m)$ un anello noetheriano locale regolare di dimensione $n$. Allora 
      \[
        n=\dh A\coloneqq\sup\set{\dhp_A M\ | \ M \text{ è } A\text{-modulo }}
      \]
     \end{teo}

     \begin{oss}
      Sappiamo che esiste un $A$ modulo, cioè $k$, tale che $\dhp k =n$, cosicché $\dh A \geq n$.
      Per il Corollario \ref{cor:regfingen}, inoltre, se $M$ è finitamente generato $\dhp k \leq n$; se mostriamo l'ipotesi di finitezza è
      superflua allora abbiamo la tesi.
     \end{oss}
     
     Enunciamo due lemmi più generali che ci serviranno per la dimostrazione del teorema:
     
     \begin{lemma}
      Se $\Ext^1(\faktor A I, X)=0$ per ogni ideale $I\subseteq A$, allora $X$ è iniettivo.
     \end{lemma}

     \begin{proof}
      Supponiamo di avere un morfismo iniettivo $g$ e $f$ come segue
     \begin{center}
      \begin{tikzpicture}[scale=1.5]
		\node (a) at (-1,0) {$0$};
		\node (b) at (0,0) {$N$};
		\node (c) at (1,0) {$M$};
		\node (d) at (0,-1) {$X$};
		\draw[->] (a) -- (b);
		\draw[->] (b) to node[left]{$f$} (d);
		\draw[->] (b) to node[above]{$g$} (c);
		\draw[->,dashed] (c) -- (d);
      \end{tikzpicture}
     \end{center}
     Consideriamo la famiglia delle possibili estensioni di $f$
     \[
      \mathcal{F}=\set{(N',f')\ | \ N\subset N'\subset M ,\ f'\colon N'\rightarrow X \text{ e } f'\lvert_N=f}
     \]
     $\mathcal{F}\neq\varnothing$ e le catene ammettono maggiorante rispetto all'ordinamento $(N',f')<(N'',f'')$ se e solo se
     $N'\subseteq N''$ e $f''\lvert_{N'}=f'$. Allora per il Lemma di Zorn esiste almeno un elemento massimale $(N',f')$.
     \begin{center}
      \begin{tikzpicture}[scale=1.5]
		\node (a) at (-1,0) {$0$};
		\node (b) at (0,0) {$N$};
		\node (c) at (1,0) {$N'$};
		\node (c1) at (2,0) {$M$};
		\node (d) at (0,-1) {$X$};
		\draw[->] (a) -- (b);
		\draw[->] (b) to node[left]{$f$} (d);
		\draw[->] (c) to node[left]{$f'$} (d);
		\draw[->] (b) to node[above]{$ $} (c);
		\draw[->] (c) to node[above]{$ $} (c1);
% 		\draw[->,dashed] (c1) -- (d);
      \end{tikzpicture}
     \end{center}
     Supponiamo che $N'\neq M$, ossia esiste $m\in M\setminus N'$, e mostriamo che in tal caso possiamo costruire un'estensione di
     $f'$ a $\langle N',m \rangle$.\\
     Sia $I=\set{a\in A \colon \ am\in N'}$ e $\mu\colon A\rightarrow M$ 
     la moltiplicazione per a destra per $m$; per costruzione $\mu(I)=N'$, invece chiamiamo $N''=\mu (A)$. Vogliamo mostrare che
     $f'\circ \mu$ si estende anch'essa a tutto $A$. A tale scopo consideriamo la sequenza esatta
      \begin{center}
	\begin{tikzpicture}[scale=1]
	 \node (a) at (-1,0) {$0$};
 	 \node (b) at (0,0) {$I$};
	 \node (c) at (1,0) {$A$};
 	 \node (d) at (2,0) {$\faktor A I$};
 	 \node (a3) at (3,0) {$0$};
	 \draw[->] (a) -- (b);
	 \draw[->] (b) to node[above]{$ $} (c);
	 \draw[->] (c) to node[above]{$ $} (d);
	 \draw[->] (d) -- (a3);
       \end{tikzpicture}
      \end{center} 
      e applichiamo $\Hom(\_,X)$. Per ipotesi otteniamo una sequenza esatta
       \begin{center}
	\begin{tikzpicture}[scale=2.3]
	 \node (a) at (-0.8,0) {$0$};
 	 \node (b) at (0,0) {$\Hom(I,X)$};
	 \node (c) at (1,0) {$\Hom(A,X)$};
 	 \node (d) at (2,0) {$\Hom(\faktor A I,X)$};
 	 \node (a3) at (2.8,0) {$0$};
	 \draw[->] (a) -- (b);
	 \draw[->] (b) to node[above]{$ $} (c);
	 \draw[->] (c) to node[above]{$ $} (d);
	 \draw[->] (d) -- (a3);
       \end{tikzpicture}
      \end{center} 
      e grazie alla suriettività  esiste $h=f'\circ\mu\in\Hom(A,X)$. Possiamo definire quindi $f''\colon \langle N',m \rangle \rightarrow X$
      come $n'+ am\mapsto f'(n')+ g(a)$. Il morfismo è ben definito, preso infatti $\bar n +\bar a m=n'+ am$ allora
      $n'-\bar n= (\bar a - a)m\in N'$ cosicché $g(\bar a-a)=f'((\bar a - a)m)= f'(n'-\bar n)$. Chiaramente $f''\lvert_N=f$ e è un morfismo
      di $A$ moduli, dunque è un'estensione. L'unica possibilità è quindi che $N'=M$ e quindi $X$ è iniettivo.
     \end{proof}
     
     Ricordiamo che 
     \[
      \dhp_A(M)\coloneqq \sup\set{n\colon \exists N \ \text{ tale che } \Ext_A^{n}( M, N)\neq0 }
     \]

      \begin{lemma}
      Sia $A$ un anello e $I$ un suo ideale. Se $\dhp_A\left( \faktor{A}{I}\right)\leq n$, allora per ogni $A$ modulo
      $\dhp_A M\leq n$.	
     \end{lemma}
     \begin{proof}
     Consideriamo un modulo $Y$ qualsiasi, per ipotesi $$\Ext^{n+1}(\faktor A I, Y)=0.$$
     Se esiste 
     \[
      0\rightarrow Y \rightarrow I^0 \rightarrow I^1 \rightarrow \dots \rightarrow I^{n-1} \rightarrow X \rightarrow 0
     \]
     esatta con $I^i$ iniettivi, allora $\Ext^{n+1}(\faktor A I, Y)=\Ext^{1}(\faktor A I, X)=0$ , cioè anche $X$ è iniettivo.
     In particolare dato che la dimensione comologica iniettiva di un modulo è $k$ se e solo se esiste
     \[
      0\rightarrow Y \rightarrow I^0 \rightarrow I^1 \rightarrow \dots \rightarrow I^{k} \rightarrow 0
     \]
      esatta di lunghezza minima e con $I^i$ oggetti iniettivi\footnote{Si dimostra come nel caso dei proiettivi.}, allora ogni
      $Y$ ha risoluzioni libere iniettive lunghe al più $n$.\\
      Per definizione dire che per ogni $Y$ $\dhi Y\leq n$ è come dire che $\Ext_A^i(M,Y)=0$ per ogni $i>n$, che a sua volta
      è equivalente al fatto che $\dhp_AM\leq n$.     
    
     \end{proof}
     
     Per provare il teorema ci siamo ridotti a dimostrare che 
     
     \begin{lemma}
      Sia $(A,\m)$ un anello noetheriano locale regolare di dimensione $n$. 
      Allora  per ogni $I$ $\dhp_A\left( \faktor{A}{I}\right)\leq n$.
     \end{lemma}

     \begin{proof}
      $\faktor{A}{I}$ è un $A$-modulo finitamente generato dato che $A$ è un anello noetheriano. La tesi è
      dunque ovviamente vera per il Corollario \ref{cor:regfingen}.
     \end{proof}

     Mostriamo ora che vale anche il viceversa:
     
     \begin{teo}
      Sia $(A,\m)$ un anello noetheriano locale. Se $r=\dh A<\infty$ allora $A$ è regolare.
     \end{teo}

     \begin{proof}
      Se $M$ è finitamente generato $\dhp_AM\leq r$, quindi $k$ ha una risoluzione libera proiettiva lunga al più $r$.
      Dato che i $Tor$ controllano tutte le lunghezza in effetti ogni modulo allora ha una risoluzione libera proiettiva
      lunga al più $r$. \\
      Sia $n=\dim A$ e $x_1,\dots, x_s$ un insieme di genetarori di $\m$. Per Nakayama allora le loro classi sono anche una
      $k$-base di $\m/\m^2$. Mostriamo per induzione su $s$ la tesi.\\
      $s=0$ Allora $A$ è un campo, che è regolare per definizione.\\
      $s>0$ Procediamo per passi.\\

	      \begin{passo}[\textbf{1}]
	       Esiste $x\in \m/\m^2$ che non è un divisore di zero.
	      \end{passo}

	      Ricoridiamo che
	      $\mathcal{D}(A)\cup \set{0}=\cup^t_{i=1}P_i$ con $P_i$ primi associati; per il teorema di unicità della decomposizione
	      primaria per ogni $i$ esiste $b_i$ tale che $\Ann (b_i)=P_i$. Vogliamo mostrare che $ \m\nsubseteq \m^2 \cup^t_{i=1}P_i$.
	      Per Nakayama $\m\nsubseteq \m^2$, altrimenti sarebbe zero, invece se $ \m\subseteq\cup^t_{i=1}P_i$
	      per il lemma di scansamento dovrebbe coincidere con uno dei primi; una terza possibilità è che $\m^2\cup^k_{i=1}P_i$ 
	      sia l'unione minimale che lo contiene (a meno di rinominare i primi), allora  prendiamo 
	      $y_0\in \m \setminus\cup^k_{i=1}P_i$  e $y_j\in\m\setminus(\m^2\cup^k_{i=1,i\neq j}P_i)$, l'elemento $y_1+y_0y_2\cdots y_k\in \m$
	      ma ciò è assurdo perché per la scelta degli $y_i$ questa somma non può stare in $\m$.  Perciò l'unica possibilità è $\m=P=\Ann a$.
	      Per ipotesi esiste 
	      \[
	      \dots\xrightarrow{\partial^{-4}}F^{-3}\xrightarrow{\partial^{-3}} F^{-2}\xrightarrow{\partial^{-2}} F^{-1}\xrightarrow{\partial^{-1}}F^0\xrightarrow{\partial^{0}} k\rightarrow 0
	      \]
	      risoluzione libera minimale di $k$ lunga $r$, allora $i(F^{-r})\subseteq \m F^{-r+1}$.
	      Allora $ai(F^{-r})\subseteq a\m F^{-r+1}=0$ che è assurdo poiché $F^{-r}$ è libero. 
	      Perciò  $ \m\nsubseteq \m^2 \cup^t_{i=1}P_i$.\\

			\begin{flushright}
	                 \checkmark
	                \end{flushright}	      
	      
	      \begin{passo}[\textbf{2}]
	       Sia $B=A/(x)$. Possiamo ridurci a dimostrare che $\dh B<\infty$.
	      \end{passo}

	      Sappiamo che $\dim B=\dim A-1=n-1$ e $B$ locale noetheriano con massimale $\m_B=\m/(x)$. Dato che $\bar x\neq 0$ in $\m/\m^2$,
	      possiamo\footnote{Le basi di uno spazio vettoriale hanno tutte la stessa cardinalità.} supporre $x=x_1$.
	      $x_1,x_2,\dots, x_s$  sono una base di $\m/\m^2$ come $k$ spazio vettoriale
	      e $\m_B$ sarà generato da $\pi( x_2),\dots, \pi (x_s)$. Se dimostriamo che $\dh B<\infty$ ricaviamo, induttivamente, che $B$ è regolare e
	      quindi $n-1=s-1$, che ci da anche $n=s$, cioè $A$ è regolare.

			\begin{flushright}
	                 \checkmark
	                \end{flushright}	      
	      
	      \begin{passo}[\textbf{3}]
	       Se $\dhp_B\m_B$ è finita allora $\dh B<\infty$.
	      \end{passo}
	      
	      Basta osservare che sostituendo a $\m_B$ una sua risoluzione libera finita
	      nella successione esatta
	       \begin{center}
	       \begin{tikzpicture}[scale=1]
	       \node (a) at (-1,0) {$0$};
	       \node (b) at (0,0) {$\m_B$};
	       \node (c) at (1,0) {$B$};
	       \node (d) at (2,0) {$k$};
	       \node (a3) at (3,0) {$0$};
	       \draw[->] (a) -- (b);
	       \draw[->] (b) to node[above]{$ $} (c);
	       \draw[->] (c) to node[above]{$ $} (d);
	       \draw[->] (d) -- (a3);
	       \end{tikzpicture}
	       \end{center}
	      troviamo una risoluzione libera di $k=B/\m_B=A/\m$ e dunque $\dhp_B k<\infty$.
	      
			\begin{flushright}
	                 \checkmark
	                \end{flushright}
	      
	      \begin{passo}[\textbf{4}]
	       $\m_B$ è un fattore diretto di $\faktor \m {x\m}$ e quindi basta mostrare che 
	       $\dhp_B\faktor \m {x\m}$ è finita.
	      \end{passo}

              Diciamo che $\m_B=\faktor {\m} {(x)} $ è un fattore diretto di $\faktor \m {x\m}$ sia come 
              $A$ che come $B$ modulo. Infatti,
                consideriamo la mappa  suriettiva
	      \[
	       \Phi\colon \faktor \m {x\m}\twoheadrightarrow \faktor {\m} {(x)} 
	      \]
	      e mostriamo che esiste una sezione, in tal caso avremmo  $\faktor \m {x\m}= \faktor {\m} {(x)} \oplus Z$.\\
	      Vale che
	      \[
	       \faktor {\m} {(x)}= \faktor {(x_2,\dots, x_s)+(x)} {(x)} =\faktor{(x_2,\dots, x_s)}{(x)\cap (x_2,\dots, x_s)}
	      \]
	      Possiamo definire allora $s\colon x_i\mapsto x_i$ per $i=2,\dots, n$. La mappa $s$ è ben definita, siamo
	      $y\in(x)\cap (x_2,\dots, x_s)$ allora 
	      \[
	       y=\sum_{i=2}^sf_ix_i=-f_1x
	      \]
	      quindi $\sum_{i=1}^sf_ix_i=0$ in $\m$ e perciò anche in $\m/\m^2$. Abbiamo quindi una combinazione lineare a coefficienti
	      in $k$ degli $x_i$ che è nulla, ma poiché sono una base gli $\bar f_i=0$, ossia $f_i\in\m$ per ogni $i$. In particolare
	      allora $y=-f_1x\in x\m$. 	      
			\begin{flushright}
	                 \checkmark
	                \end{flushright}
	      
	      \begin{passo}[\textbf{5}]
	       $\dhp\faktor \m {x\m}$ è finita.
	      \end{passo}
         
	      Diciamo che $\dhp_B\faktor \m {x\m}<\infty$ e per dimostrarlo costruiamo una sua risoluzione libera lunga al più $r$.
	      Sappiamo per ipotesi che esiste una risoluzione libera di $\m$ come $A$ modulo lunga al più $r+1$
	      \[
	      0\rightarrow F^{-r}\xrightarrow{\partial^{-r}}\dots\xrightarrow{\partial^{-2}} F^{-1}\xrightarrow{\partial^{-1}}F^0\rightarrow 0
	      \]
	      con $F^{-i}=A^{n_i}$. Osserviamo che $\m \otimes B =\m\otimes\faktor A {(x)} = \faktor \m {x\m}$, allora se applicando 
	      $\_\otimes B$ a tale complesso ottenessimo che    
	      \[
	      0\rightarrow B^{n_k}\xrightarrow{\partial^{-k}}\dots\xrightarrow{\partial^{-2}} B^{n_1}\xrightarrow{\partial^{-1}}B^{n_0}\rightarrow 0
	      \]
	      è aciclico avremmo la risoluzione che ci serve per concludere. Tuttavia la comologia di questo complesso è 
	      data proprio dai $\Tor^i(\m,B)$, che sono effettivamente zero per $i>0$. Infatti possiamo calcolarli a partire dalla risoluzione di $B$
	      \[
	       0\rightarrow A\xrightarrow{\cdot x} A\rightarrow 0
	      \]
	      Chiaramente allora $\Tor^i(\m,B)=0$ per $i>1$. Inoltre tensorizzando per $\m$ otteniamo
	      \[
	       0\rightarrow \m\xrightarrow{\cdot x} \m\rightarrow 0
	      \]
	      ma, per la scelta di $x$, questa mappa è iniettiva e dunque $\Tor^1(\m,B)=0$.
     \end{proof}

     

     \begin{cor}
      Se $A$ è un anello noetheriano locale regolare $p\in\Spec A$, allora $A_p$ è regolare.
     \end{cor}
     \begin{proof}
      $A\supset \m\supset p$ e consideriamo $A_p$ come $A$ modulo. Allora ammette una risoluzione libera  
      \[
	  0\rightarrow F^{-n}\xrightarrow{\partial^{-n}}\dots\xrightarrow{\partial^{-2}} F^{-1}\xrightarrow{\partial^{-1}}F^0\rightarrow 0
      \]
      Applichiamo $\_\otimes A_p $
      \[
	  0\rightarrow \inverti S F^{-n}\xrightarrow{\partial^{-n}}\dots\xrightarrow{\partial^{-2}}\inverti S F^{-1}\xrightarrow{\partial^{-1}}\inverti S F^0\rightarrow 0
      \]
      per piattezza questa rimane esatta ed è quindi una risoluzione libera del campo residuo $k=\inverti{S}(A/p)$.
     \end{proof}

      
     \begin{oss}
      Se $\dim A_p=1$ e $A$ è un dominio, allora $A_p$ è un anello di valutazione discreta.
     \end{oss}

     
     \section{Anelli graduati}
     Studiando la dimensione di un anello noetheriano locale abbiamo incontrato questo risultato che correla gli anelli regolari 
     con gli anelli graduati:
     
     \begin{teo}
      Sia $(A,\m)$ un anello noetheriano locale tale che $x_1,\dots, x_{m}$ siamo una $k$-base di $\faktor{\m}{\m^2}$.
      Allora $A$ è un anello regolare se e solo se $Gr_{\m}(A)$ è isomorfo a $k[u_1,\dots, u_{m}]$ come anelli graduati.
     \end{teo}

     Possiamo estendere quanto detto nel paragrafo precedente all'anello graduato $S=k[x_1,\dots, x_{n}]$, considerando 
     però solo i moduli graduati. Ricordiamo che un $S$ modulo libero graduato $(S[k])^n=S^{k+n}$.\\
     Valgono con alcune accortezza allora i seguenti fatti:
     \begin{itemize}
      \item La lunghezza e la dimensione di una risoluzione libera minimale di un modulo $M$ è controllata dai $Tor^i(k,M)$
	    dove $k$ è lo $S$ modulo $S/S^+$.
      \item $x_1,\dots, x_{n}$ sono una successione regolare.
      \item (\emph{Teorema di Hilbert}) Ogni modulo finitamente generato  ha una risoluzione lunga al più $n$.
     \end{itemize}


 \chapter{Moduli piatti e proiettivi}
    Mostriamo adesso alcuni risultati sui moduli ottenibili grazie alla teoria sviluppata sulla comologia.
   Ricordiamo alcuni fatti:
   \begin{prop}\*
    \begin{enumerate}
     \item Se $M$ è un $A$ modulo piatto, allora Se $\inverti S M$ è un $\inverti S A$ modulo piatto.
     \item $M$ è un $A$ modulo piatto se e solo se $M_{\m}$ è piatto per ogni $\m\in \Spec(A)$  massimale.
    \end{enumerate}
   \end{prop}
   
   \begin{proof}
   1. Basta osservare che un $\inverti S A$ modulo $X$ è anche un $A$ modulo e che 
   \[ X \otimes_{\inverti S A}\inverti S M\simeq  X \otimes_{\inverti S A} (\inverti S A \otimes_A M)\simeq
      X \otimes_A M\]
   2. C'è da dimostrare solo che la condizione è necessaria. Consideriamo una successione di $A$ moduli esatta
   \[
    0\rightarrow X\rightarrow Y
   \]
    vogliamo mostrare che in 
    \[
    0\rightarrow \ker f \rightarrow M \otimes X\xrightarrow{f} M\otimes Y
   \]
   $\ker f=0$. Sappiamo che
   \[
    0\rightarrow X_{\m}\rightarrow Y_{\m}
   \]
   è esatta e per ipotesi allora lo è anche 
   \[
    0\rightarrow M_{\m}\otimes X_{\m}\rightarrow M_{\m}\otimes Y_{\m}
   \]
   Ma $M_{\m}\otimes B_{\m}=(M\otimes B)_{\m}$ e quindi
    \[
    0\rightarrow (M\otimes X)_{\m}\rightarrow (M\otimes Y)_{\m}
   \]
   è esatta cosicché $\ker f_{\m}=0$.\\
   Vale essere zero è una proprietà locale.Infatti 
   supponiamo che un modulo $K\neq0$ sia tale che $K_{\m}=0$ per ogni $\m\in \Spec(A)$  massimale, 
   allora esistono un elemento $0\neq k\in K$ e un massimale $\m\supseteq\Ann(k)$; localizzando 
   avremmo che $\frac{k}{1}=0$ e quindi esisterebbe $s\in A\setminus \m$ tale che $sk=0$, ma
   per definizione $s\in I$. Assurdo.\\
   Allora $\ker f=0$ e quindi $M$ è piatto.
   \end{proof}


   
   \begin{lemma}\label{lemma:torquozienti}
    Sia $A$ un anello commutativo con identità e $M$ un $A$ modulo. 
    $M$ è piatto se e solo se per ogni ideale $I\subseteq A$ $$\Tor_A^1 ( \faktor{A}{I}, M )=0.$$
    \end{lemma}

   \begin{proof}
    Ovviamente se $M$ è piatto $\Tor_A^1 ( \_, M )=0$. Mostriamo ora che questa condizione è sufficiente.
    Consideriamo una successione esatta corta
    \begin{center}
    \begin{tikzpicture}[scale=1]
	 \node (a) at (-1,0) {$0$};
 	 \node (b) at (0,0) {$X$};
	 \node (c) at (1,0) {$Y$};
 	 \node (d) at (2,0) {$Z$};
 	 \node (a3) at (3,0) {$0$};
	 \draw[->] (a) -- (b);
	 \draw[->] (b) to node[above]{$a $} (c);
	 \draw[->] (c) to node[above]{$ $} (d);
	 \draw[->] (d) -- (a3);
       \end{tikzpicture}
      \end{center}
      e tensorizziamo per $M$
    \begin{center}
    \begin{tikzpicture}[scale=1.8]
	 \node (a) at (-1,0) {$ $};
 	 \node (b) at (0,0) {$X\otimes M$};
	 \node (c) at (1,0) {$Y\otimes M$};
 	 \node (d) at (2,0) {$Z\otimes M$};
 	 \node (a3) at (2.8,0) {$0$};
% 	 \draw[->] (a) -- (b);
	 \draw[->] (b) to node[above]{$b $} (c);
	 \draw[->] (c) to node[above]{$ $} (d);
	 \draw[->] (d) -- (a3);
       \end{tikzpicture}
      \end{center}
      Se $b$ è iniettiva allo abbiamo la tesi.
    Pocediamo per passi.
    \begin{itemize}
     \item Possiamo assumere che $Y$ sia finitamente generato. Supponiamo per assurdo che esistano $x_i\in X$ e $m_i\in M$ tali che
	\[
	 \sum x_i\otimes m_i\neq 0 
	\]
	\[
	 \sum a(x_i)\otimes m_i= 0 
	\]
	Ricordiamo che 
	\[
	Y\otimes M= \frac{ \oplus Ae_{y,m}}{Rel}
	\]
	dove $Rel$ è l'insieme delle relazioni di equivalenza sugli elementi $e_{y,m}$. Chiamiamo $Y'$ il modulo generato dagli
	$a(x_i)$ e da tutti gli altri elementi che nell'insieme $Rel$ compaiono in relazione con gli $e_{a(x_i),m}$.
	Allora $\sum a(x_i)\otimes m_i= 0$ è zero anche in $Y'\otimes M$. 
	Chiamiamo ora $X'$ il modulo generato dagli $x_i$ e $a'\colon X'\rightarrow Y'$ la restrizione di $a$, che quindi deve essere
	iniettiva. Supponiamo di aver dimostrato che $b$ è iniettiva nel caso finitamente generato,
	allora  $\sum x_i\otimes m_i =0 $ in $X'\otimes M$ e visto che le mappe di moduli portano zero in zero, usando l'immersione
	di $X'\otimes M$ in $X\otimes M$ abbastanza che $\sum x_i\otimes m_i =0 $.
	\item Se $Y$ è finitamente generato allora $Z$ è finitamente generato: per ipotesi $Z\simeq Y/a(X)$.
	\item $b$ è iniettiva. $Z=\langle z_1,\dots , z_n\rangle_A$, possiamo allora scrivere una successione di moduli:
		\[\begin{cases}
		   Z_0=Z_1= \langle z_1\rangle\\
		   Z_j=\langle z_1,\dots , z_j\rangle
		  \end{cases}\]
	Dato che $Z_j/Z_{j-1}=\langle z_j\rangle $ per ogni $j$ abbiamo un sequenza esatta corta
		\[\begin{matrix}
		 0\rightarrow I_j \rightarrow & A &\rightarrow & \faktor {Z_j} {Z_{j-1}} &\rightarrow 0\\
					       & a &\mapsto & az_j
		 \end{matrix}
		\]
	e dunque $Z_j/Z_{j-1}\simeq A/I_j$.\\
	Per provare la tesi mi basta mostrare che $\Tor_A^1(Z,M)=0$. In particolare mostriamo che $\Tor_A^1(Z_j,M)=0$ per ogni $j$.\\
	Se $j=1$ $\Tor_A^1(Z_1,M)=\Tor_A^1 (\faktor{A}{I_1}, M )=0$.\\
	Se $j>1$ Applichiamo il funtore $\_ \otimes M$ a 
		\[
		 0\rightarrow Z_{j-1} \rightarrow  Z_j \rightarrow  \faktor {A} {I_{j}} \rightarrow 0
		\]
	e otteniamo che  $\Tor_A^1 (Z_{j}, M )=0$, infatti
	\begin{equation*}
	\begin{matrix}
	\dots \rightarrow  Z_{j-1} \otimes M \rightarrow Z_j \otimes M \rightarrow \faktor {A} {I_{j}} \otimes M \rightarrow 0\\
	\dots\rightarrow 0 \rightarrow \Tor^1(Z_j,M) \rightarrow 0 \rightarrow  \dots
	\end{matrix}
      \end{equation*}
	dato che $\Tor_A^1 (\faktor{A}{I_j}, M )=0$ per ipotesi mentre $\Tor_A^1 (Z_{j-1}, M )=0$ per ipotesi induttiva.\\
     \end{itemize}   
   \end{proof}
  
  \begin{de}
   La \emph{torsione} di un modulo $M$ è l'insieme 
   \[
    \tors(M)\coloneqq \set{m\in M \ |\ \Ann (M) \neq 0}
   \]
  \end{de}
  \begin{lemma}
   $\tors(M)$ è un gruppo. Se $A$ è un dominio è anche un modulo.
  \end{lemma}
  \begin{lemma}
   Siano $A$ dominio e $M$ un modulo piatto. Allora $\tors(M)=0$
  \end{lemma}
  \begin{proof}
   Se $a\neq 0$ allora è ben definita la successione esatta
   		\[
		 0\rightarrow A \xrightarrow a  A  \rightarrow  \faktor {A} {(a)} \rightarrow 0
		\]
  Tensorizzando per $M$ abbiamo
		\[
		 0\rightarrow M \xrightarrow a  M  \rightarrow  \faktor {M} {(a)M} \rightarrow 0
		\]
  In particolare la moltiplicazione per $a$ che era iniettiva perché $A$ è un dominio rimane iniettiva, grazie alla piattezza,
  e quindi $\tors(M)=0$. 
  \end{proof}
  \begin{lemma}
    Siano $A$ dominio e $M$ un modulo con $\tors(M)=0$. Allora
    \[
     \Tor_A^1(\faktor {A} {(a)}, M)=0.
    \]
  \end{lemma}
  \begin{proof}
   Se $a=0$, dato che $A$ è libero, $\Tor_A^1( {A}, M)=0$. Se $a\neq0$, consideriamo la successione del lemma precedente 
		\[
		 0\rightarrow A \xrightarrow a  A  \rightarrow  \faktor {A} {(a)} \rightarrow 0
		\]
  e tensorizziamo per $M$ abbiamo
		\[
		 \cdots\rightarrow\Tor_A^1(\faktor {A} {(a)}, M)\rightarrow M \xrightarrow a  M
		 \rightarrow  \faktor {M} {(a)M} \rightarrow 0
		\]
    Ma $\ker (\cdot a)=0$ perché che il modulo è senza torsione e visto che 
		\[
		 \Tor_A^1( {A}, M)\rightarrow \Tor_A^1(\faktor {A} {(a)}, M)  \rightarrow  \ker (\cdot a)
		\]
    è esatta, allora $\Tor_A^1(\faktor {A} {(a)}, M)=0$.
  \end{proof}
  
  \begin{cor}
   $A$ {\sc pid} e e $M$ un modulo con $\tors(M)=0$. Allora $M$ è piatto.
  \end{cor}
  \begin{proof}
   $A$ {\sc pid} implica che per ogni $I$ esiste $a$ tale che $I=(a)$, allora per ogni ideale
   $\Tor_A^1(\faktor {A} {I}, M)=0$ e dunque per il Lemma \ref{lemma:torquozienti} $M$ è piatto.
  \end{proof}

  \begin{esem}\*
   \begin{itemize}
    \item $\mathbb Q$ è uno $\Z$ modulo che è piatto ma non proiettivo (né libero).
    \item $\Z\times \Z\times\dots \Z\times \dots$ è senza torsione e non è libero.
   \end{itemize}
  \end{esem}
  \begin{oss}
   Proiettivo implica sempre piatto, lo conferma che per $i>0$ per un moduli proiettivo i $\Tor$ sono tutti nulli.
  \end{oss}
  
  \begin{esem}
   $A=\mathbb C[x,y]$ e $M=(x,y)$ \marginpar{rivedi..}
  \end{esem}

  {\sc vedi fine lezione}



\end{document}
